\chapter{A Metabolites Example}
\label{A Metabolites Example}

\section{Foo section}
A metabolite is small structure that is the byproduct of the body's metabolism. To better understand the path of metabolites, an UKF was utilized for biomedical pathway modeling. Data regarding metabolites is highly influenced by noise, which is an UKF can handle while other approaches, such as regression and annealing, fail. Since the model is complex, considering that there are many unknown parameters and many metabolites in the system, utilizing an UKF is more effective than an EKF. Initial data contained information about metabolites at various time steps.
\begin{center}
    
\centering
\begin{tabular}{ |p{2cm}||p{5cm}|p{3cm}| }
    \hline
    \multicolumn{3}{|c|}{Variables in the Unscented Kalman Filter } \\ 
    \hline
    Variable & Description & Dimensions \\
    \hline
    x & Vector containing state variables & $d_x \times 1 $\\ 
    $\chi $& Sigma Point Matrix &$ d_x \times (2 d_x + 1) $\\
    F & State Transition Matrix  & $d_x \times d_x $  \\ 
    H & Observation Matrix & $d_y \times d_x$\\
    G & Input Matrix & $d_x \times d_u$\\
    u & Input Vector  & $d_u \times 1$\\
    P & Covariance matrix & $d_x \times d_x $  \\
    k & Time step  & $1 \times 1$\\
    w & Parameter vector & $d_q \times 1$\\
    v & Process Noise Vector & $d_x \times 1$\\
    $\alpha$ & Controls spread of sigma points & 1 \\
    $\beta$ & Adjust sigma point weight & $0 \leq  \beta$ \\
    $\kappa $ & Sigma point weighting constant & $3 - d_q = -14 $ \\
    $\lambda $ & Scaling parameter for weights & 1 \\
    \hline
\end{tabular}
\end{center}

\begin{enumerate}
    \item Initialize state vector and covariance
    \begin{align}
        \hat{x}_{0} = \mathbb{E}[x_{0}] = \begin{bmatrix}
           4 \\
           1 \\
           3 \\
           4
         \end{bmatrix} 
    \end{align}
    \begin{align}
        P_{x_{0}} = \mathbb{E}[(x_{0}-\hat{x}_{0})(x_{0}-\hat{x}_{0})^{T}] = \begin{bmatrix}
           .1 & 0 & 0 & 0 \\
           0 & .1 & 0 & 0 \\
           0 & 0 & .1 & 0 \\
           0 & 0 & 0 & .1
         \end{bmatrix} 
    \end{align}
    \item Calculate sigma points
    \begin{align}
        \chi_{\ 0,k-1} &= \hat{x}_{k-1} \\
        \chi_{\ i,k-1} &= \hat{x}_{k-1} + (\sqrt{(n_{x}+\lambda_{x})P_{x_{k-1}}})_{i}, &  i=1,\dots,n_{x} \\
        \chi_{\ i,k-1} &= \hat{x}_{k-1} - (\sqrt{(n_{x}+\lambda_{x})P_{x_{k-1}}})_{i},  &  i=n_{x}+1,\dots,2n_{x},
    \end{align}
    \item Calculate the weights for each sigma point \\ \\
    Weights can have positive or negative values, but will ultimately sum to 1 \cite{article6}
    \begin{align}
        \lambda_{x} = \alpha_{x}^{2}(n_{x}+\kappa_{x})-n_{x} \\
        W^{(m)}_{x,0} = \frac{\lambda_{x}}{n_{x}+ \lambda_{x} } \\
        W^{(c)}_{x,0} = \frac{\lambda_{x}}{n_{x}+ \lambda_{x} } + (1 - \alpha^{2}_{x} + \beta_{x}) \\
        W^{(m)}_{x,i} = W^{(c)}_{x,i} = \frac{\lambda_{x}}{2(n_{x}+ \lambda_{x}) }, & i=1,\dots,2n_{x} \\
        W^{\text{aug,}(m)}_{x,0} = \frac{\lambda_{x}}{2n_{x}+ \lambda_{x} } \\
        W^{\text{aug,}(c)}_{x,0} = \frac{\lambda_{x}}{2n_{x}+ \lambda_{x} } + (1 - \alpha^{2}_{x} + \beta_{x}) \\
        W^{\text{aug,}(m)}_{x,i} = W^{\text{aug,}(c)}_{x,i} = \frac{\lambda_{x}}{2(2n_{x}+ \lambda_{x}) }, &  i=1,\dots,4n_{x} \\
    \end{align}
    \item Perform a non-linear transformation on the sigma points
    \begin{align}
        \hat{w}_{k} &= \hat{w}_{k-1} \\
        \chi^{*}_{\ i,k|k-1} &= f(\chi_{\ i,k-1},\hat{w}_{k-1})
    \end{align}
    \item Calculate the mean and covariance of the transformed sigma points
    \begin{align}
        \hat{x}^{*}_{\ k|k-1} &= \sum_{i=0}^{2n_{x}} W^{(m)}_{x,i} \chi^{*}_{\ i, k|k-1} \\
        P_{x_{k|k-1}} &=  \sum_{i=0}^{2n_{x}} W^{(c)}_{x,i} (\chi^{*}_{\ i, k|k-1} - \hat{x}^{*}_{\ k|k-1})(\chi^{*}_{\ i, k|k-1} - \hat{x}^{*}_{\ k|k-1})^{T} + R_{v} % Q_k is same as R_{v} (Process Noise Covariance)
    \end{align}
    \item Re-calculate sigma points
    \begin{align}
        \chi_{\ i, k|k-1} = \chi^{*}_{\ i,k|k-1}, & i=0,...,2n_{x} \\
        \chi_{\ i, k|k-1} = \chi^{*}_{\ 0,k|k-1} + (\sqrt{(n_{x}+\lambda_{x})P_{x_{k|k-1}}})_{i},  & i=2n_{x}+1,\dots,3n_{x} \\
        \chi_{\ i, k|k-1} = \chi^{*}_{\ 0,k|k-1} - (\sqrt{(n_{x}+\lambda_{x})P_{x_{k|k-1}}})_{i}, &  i=3n_{x}+1,\dots,4n_{x}
    \end{align}
    \item Generate prediction
    \begin{align}
        \hat{x}_{k|k-1} = \sum_{i=0}^{4n_{x}} W^{\text{aug,}(m)}_{x,i} \chi_{\ i, k|k-1} \\
        \mathcal{Y}_{k|k-1} = h(\chi_{k|k-1},\hat{w_{k}}) \\
        \hat{y}_{k|k-1} = \sum_{i=0}^{4n_{x}} W^{\text{aug,}(m)}_{x,i} \mathcal{Y}_{\ i, k|k-1}
    \end{align}
    \item Calculate Kalman Gain
    \begin{align}
        P_{y_{k}} = \sum_{i=0}^{4n_{x}} W^{\text{aug,}(c)}_{x,i} (\mathcal{Y}_{i, k|k-1} - \hat{y}_{k|k-1})(\mathcal{Y}_{i, k|k-1} - \hat{y}_{k|k-1})^{T} + R_{n}  \\ %R_k is the same as R_{n} (Measurement Noise Covariance)
        P_{x_{k}y_{x}} = \sum_{i=0}^{4n_{x}} W^{\text{aug,}(c)}_{x,i} (\chi_{\ i, k|k-1} - \hat{x}^{*}_{\ k|k-1})(\mathcal{Y}_{i, k|k-1} - \hat{y}_{k|k-1})^{T} \\
        K_{k} = P_{x_{k}y_{x}}P^{-1}_{y_{k}}
    \end{align}
    \item Update Estimate and state covariance
    \begin{equation}
        P_{x_{k}} = P_{x_{k|k-1}} - K_{k}P_{y_{k}}K^{-1}_{k} 
    \end{equation}
    \begin{equation}
        \hat{x}_{k} = \hat{x}_{k|k-1} + K_{k}(y_{k}-\hat{y}_{k|k-1}),
    \end{equation}
    
\end{enumerate}