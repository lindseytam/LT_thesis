\ProvidesClass{pom_thesis}[1/16/15]
\documentclass{pom_thesis}

\usepackage[utf8]{inputenc}
\usepackage{amsmath}
\usepackage{xcolor}
\usepackage{amssymb}
\usepackage{mathtools}
\usepackage[english]{babel}
\usepackage{hyperref}
\usepackage{multicol}
\usepackage{enumitem}
\usepackage{array}
\newcolumntype{P}[1]{>{\centering\arraybackslash}p{#1}}
\newcolumntype{M}[1]{>{\centering\arraybackslash}m{#1}}
\usepackage{cite}
\usepackage{listings}
\usepackage{csquotes}
\usepackage{subfig}
\usepackage{color} %red, green, blue, yellow, cyan, magenta, black, white
\definecolor{mygreen}{RGB}{28,172,0} % color values Red, Green, Blue
\definecolor{mylilas}{RGB}{170,55,241}
\usepackage[toc,page]{appendix}
\newcommand{\comment}[1]{}


\title{Kalman Filter Methods for Model Parametrization}\par

\author{Lindsey Tam }
\advisor{Blerta Shtylla}

\begin{document}

\maketitle

\begin{abstract}
	This thesis explores the theory and applications of Kalman Filters in both state and parameter estimation. Kalman Filters are able to recursively generate predictions for linear systems and these predictions become progressively more accurate because of the system's ability to correct predictions using incoming data. Nonlinear forms of the Kalman Filter exist, including the Extended Kalman Filter (EKF) and the Unscented Kalman Filter (UKF). The EKF works by linearizing the system around the mean, making it more effective for simpler and lower dimensional systems. The UKF addresses the shortcomings of the EKF by linearizing the system around multiple points, known as sigma points. After exploring the theory behind these techniques, this paper implements a few examples of the EKF and the UKF for both state and parameter estimation. \end{abstract}

\pagenumbering{roman}
\tableofcontents


\chapter{Introduction}
\label{Introduction}

This paper explores the Kalman Filter, Extended Kalman Filter, and the Unscented Kalman Filter. The Kalman Filter is a data prediction method that estimates the value of states in linear dynamical systems. This is a recursive predictive-corrective process that enables the Kalman Filters to continuously generate predictions about state variables without relying on large amounts of initial data.  In the case of nonlinear systems, alternative forms of the Kalman Filter were developed, including the Extended Kalman Filter (EKF) and the Unscented Kalman Filter (UKF). The EKF linearizes the nonlinear system around the mean by taking the Jacobean of the nonlinear function. Navigation and signal processing are the main applications of the EKF Instead of linearizing the system around a single point, the UKF utilizes many points, known as Sigma points. The Sigma points then undergo a nonlinear transformation known as the Unscented Transform. Applications of the UKF includes modeling biological systems. Though we will discuss all three versions, the exploration of the UKF will be the main focus of this paper.  \\ \\
In addition to understanding the theory behind these algorithms, this paper also explores applications of the UKF. An example of the UKF can be applied to the Van der Pol oscillator, which is a self sustaining nonconservative oscillator. The purpose of this simple example is to demonstrate how a simple version of UKF can be implemented on MATLAB. We will also observe how process and measurement noise influences the system. A second example includes modeling the biological pathway of metabolites, which are molecules in the human body that are the byproduct of the metabolism. This example is slightly more complex, with four different states and 18 unknown parameters. The focus of this example is to illustrate the following three things: how the UKF can correct for multiple states, working with systems in higher dimensions, and how the UKF can be applied to extract parameter values. \\ \\
The ultimate goal of researching the UKF is to create a model that can forecast glucose values in real time of patients with Type 1 diabetes. This is inspired by a separate study that was able to do real-time glucose forecasting in Type-2 diabetes patients. Using an existing model of Type 1 diabetes that includes 12 states, the UKF will be most useful in extracting parameter values. To gauge the effectiveness of the model, we will be comparing the results with mouse data. Implementing this model has the potential to improve treatment methods and the quality of life of Type 1 diabetes patients.






\chapter{State Space Models}
\label{State Space Models}


Applying the Kalman Filter to a system requires an understanding of state space models. A state space model represents a system's inputs, outputs, and states by describing a series of first order differential equations. A linear continuous state space system has a general form of 
\begin{align*}
&\dot x(t)= A x(t) + B u(t) +w(t) \quad &\text{(system)}, \\
&y(t) = H x(t) +v(t) \quad &\text{(observation)} ,
\end{align*}
where $x$ is the state vector, $A$ is a square transformation matrix, $B$ is the input matrix, $u$ is a system input, $w$ is the process noise vector, $y$ is the transformed prediction, $H$ is the observation matrix, $v$ is the measurement noise vector, and $t$ is time. \\

\noindent For implementing Kalman Filters, the continuous state space system must be discretized into time steps, call it $k$. The general equation is given below as
\begin{align*}
& x_k= e^{At}x_{k-1} + \int_0^t e^{A(t-r))} B u_k dr + w_k \quad &\text{(system)}, \\
&y_k = H x_k + v_k  &\text{   (observation)} ,
\end{align*}
where $e^{At}$ is the matrix exponential of $A$. Generally, the matrix exponential is very time-consuming to compute, especially if the matrix is not diagnolizable. In these cases, an alternative solution is to use Euler's method, which approximates the discretized system using small time steps, as shown by 
\begin{align*}
& x_k= x_{k-1} + t f(x) &\text{(system)}, \\
&y_k = H x_k + v_k  &\text{(observation)},
\end{align*}
where $t$ is some small time step and $f(x)$ is the state after it undergoes a transformation. In most of the examples that are explored in this thesis, computing the matrix exponential is inefficient and results in excessive computing time. Therefore, Euler's method will be mostly used for discretization.

\noindent Consider an example of a moving mass, call it $m$, under a force, call it $u(t)$. The goal of this system is to predict the mass's position and velocity given its previous position and velocity. Therefore, position and velocity are the two states in this system. From differential equations, recall that the position of the mass is denoted by $x(t)$, the velocity of the mass is denoted by $\dot{x}(t)$, and the acceleration is denoted by $\ddot{x}(t)$. According to Newton's second law of motion, the force exerted on an object is given by the equation $F=ma$, or
\begin{align*}
u(t)=m\ddot{x}(t).
\end{align*}

\noindent This is a second order differential equation. To convert it into its state space format, begin by transforming it into a first order differential equation by substituting $x_1(t)$ for $x(t)$ and $x_2(t)$ for $\dot{x}(t)$, resulting in
\begin{align*}
\dot{x}_1(t) = x_2(t), \\
\dot{x}_2(t) = \frac{u(t)}{m}.
\end{align*}

\noindent For the sake of simplicity, assume this system has no process or measurement noise. The state vector, $x$, containing both position and velocity, can now be rewritten as
$$
x=
\begin{bmatrix}
x_1(t)\\
x_2(t)
\end{bmatrix}.
$$

\noindent In order to predict the state of the system at the next time step, take the derivative of the state vector and rewrite it in vector-matrix form (which is possible because this system is linear), resulting in
$$
\dot x=
\begin{bmatrix}
\dot{x}_1(t)\\
\dot{x}_2(t)
\end{bmatrix}=
\begin{bmatrix}
x_2(t)\\
\frac{u(t)}{m}
\end{bmatrix}=
\begin{bmatrix}
0 & 1\\
0 & 0
\end{bmatrix} x +
\begin{bmatrix}
0\\
\frac{1}{m}
\end{bmatrix} u(t).
$$
Here, one can see how this form can be applied to the general linear state space model where $A = 
\begin{bsmallmatrix}
0 & 1\\
0 & 0
\end{bsmallmatrix} $ and $ B=[0, \frac{1}{m}]^T$. \\

\noindent For the sake of this example, assume that only the position of the mass is measurable. Therefore, the transformed prediction, $y$, is given by
$$
 y=
\begin{bmatrix}
x_1(t)\\
0
\end{bmatrix}=
\begin{bmatrix}
1 \\
0
\end{bmatrix} x=
x_1(t). $$

\noindent Now that the continuous state system is defined, it can be discretized. Since this example considers a simple linear system, the matrix exponential can be calculated as follows 

$$e^{At} =
\begin{bmatrix}
e^0& e^t \\
0 & e^0
\end{bmatrix}=
\begin{bmatrix}
1& t\\
0 & 1
\end{bmatrix},
$$

$$\int_0^t e^{A(t-r))}  e^{At} B u_k dr=
\int_0^t e^{A(t-r))}  \begin{bmatrix}
1& t\\
0 & 1
\end{bmatrix} 
\begin{bmatrix}
0\\
\frac{1}{m}
\end{bmatrix} u(t) dt =
\int_0^t e^{A(t-r))}   \begin{bmatrix}
\frac{t}{m}\\
\frac{1}{m} 
\end{bmatrix} u(t) dt =
\begin{bmatrix}
\frac{t^2}{2m}\\
\frac{t}{m}
\end{bmatrix} u(t).
$$
Now the discretized state space model can be written as 


\begin{align*}
x &=
\begin{bmatrix}
1& t\\
0 & 1
\end{bmatrix}+x_{k-1}
\begin{bmatrix}
\frac{t^2}{2m}\\
\frac{t}{m}
\end{bmatrix} u(t) &\text{(system)}, \\
y &=
\begin{bmatrix}
1 \\
0
\end{bmatrix} x &\text{(observation)}.
\end{align*} \\

\noindent It is important to be able to discretize state space models because the goal of the Kalman Filter is to make some prediction about a future state. Therefore, discretization enables the Kalman Filter to make approximations about the system's state given past information. In addition, discretization makes numerical computing more efficient.


























\chapter{Kalman Filters}
\label{Kalman Filters}


The Kalman Filter (KF) recursively generates predictions for linear dynamical systems \cite{inbook}. The basic foundations of the this algorithm include generating a prediction given some initial knowledge of the data and using actual measurements from the system to continually correct the prediction. A high level overview of this process is illustrated in Figure ~\ref{fig:coffee}, which visualizes the parallel between the KF and a coffee filter. 	Unlike other predictive methods like machine learning, the KF can begin generating estimates without large amounts of initial data. The KF begins by assuming the given data is noisy and Gaussian \cite{inproceedings, article7}. The first predictive step assumes knowledge of initial states and the model process. Since we know the system is linear, the model process can be denoted as a matrix and the states can be expressed as a vector. Through matrix vector multiplication, this algorithm simulates how states are transformed after undergoing some process. The next step involves calculating the covariance in order to calculate the Kalman Gain, which is a measure of how much the estimate should be changed given actual measurements of the system. The corrective step utilizes the Kalman Gain, which is a matrix representing a weight to correct the prediction. This process can be done recursively, allowing the model to become progressively more accurate as more data is added, as visualized in Figure ~\ref{fig:recursive}. Overtime, it is expected that the model will converge with the actual system measurements. Examples and code for implementing the KF are not explored in this thesis, but can be found in \cite{article7}
\\  \\

\begin{figure}[ht]
    \centering
    \includegraphics[scale = 0.4]{coffee.png}
    \caption{A simpler way to explore the KF is to facilitate a comparison with a coffee filter. This image is taken from \cite{article7}.}
      \label{fig:coffee}
    \end{figure}

\begin{figure}[ht]
    \centering
    \includegraphics[scale = 0.3]{diagram.png}
    \caption{This is a basic diagram taken from \cite{kohanbash_2014} that demonstrates the recursive nature of the Kalman Filter. The output of the KF is an estimation of the states in a future time step while the inputs of the KF is incoming measurements from the system. The KF is able to continually generate predictions and correct the predictions as long as system measurements are being inputted into the system.}
    \label{fig:recursive}
\end{figure}
\clearpage




\newpage

\noindent The most common application of KFs is in navigation, image processing, and finance. A relevant example is using computer vision to monitor and track vehicles in real time. This enables a traffic camera to know when to take a picture to capture a vehicle's license plate. Another example is the development of the Global Positioning Systems (GPS) \cite{lim_ong_lim_koo_2016}. The KF also has many aeronautical applications, which include long-distance flight and autopilot systems. In fact, the KF was initially created to help develop the navigation system in the Apollo Program \cite{kalmanbio}.  \\ \\


\noindent The Kalman Filter is named after its developer, Rudolph Emil Kalman. Kalman was born on May 19, 1930 in Budapest, Hungary. After arriving in the United States, Kalman completed his undergraduate studies and masters degree in electrical engineering from Massachusetts Institute of Technology and completed his doctoral degree in Columbia University. He would spend the next years of his life teaching. In the 1960s to 1970s he became a professor at Stanford university \cite{kalmanbio}. In the 1970s and 1990s, Kalman spent time as a professor of engineering at the University of Florida. Kalman is most known for his work on the Kalman Filter, which was developed in the late 1950s. The Kalman Filter greatly aided the United States' military projects, resulting in former President Obama to award Kalman the National Medal of Science in 2009. In addition, in 1985, Kalman was awarded the Kyoto Prize, which is the Japanese version of the Nobel Peace Prize. Kalman passed away July 2, 2016 at the age of 86 and is survived by his wife and two children \cite{Kalman_bio}. 

\newpage

\section{Kalman Filter Algorithm}

\noindent Recall from Chapter~\ref{chap:two} that the KF requires the system to have a discretized state space form of \begin{align*}
& x_k= e^{At}x_{k-1} + \int_0^t e^{A(t-r))} B u_k dr + w_k \quad &\text{(system)}, \\
&y_k = H x_k + v_k  &\text{   (observation)}.
\end{align*}
 This thesis will not be exploring any examples with an input function, so moving forward $x_k$ will be expressed as 
\begin{align*}
& x_k= Fx_{k-1} + w_k \quad &\text{(system)},
\end{align*}
where $F=e^{At}$ and represent the transformation of the state function, and $w_k = \mathcal{N}(0, c_1)$ or Gaussian white noise. This formula means that the prediction of the next state is equal to a transformation of the previous state with some additive process noise. This general equation is significant to the generation of predictions and will continue to be used in different versions of the KF. \\ 

\noindent Another important aspect of the KF is the incorporation of state measurements. It cannot be assumed that there will always be measurements for all states. More likely, there will only be measurements for a subset of states, Therefore, it is necessary to transform the prediction into a format that can be compared with state values. The transformed estimate of the system measurement at the next time step, call it $y_{k}$, is given by 
\begin{align*}
	y_{k} = H x_{k} + v_{k} \quad &\text{(observation)} ,
\end{align*}
where $H$ is the matrix observation function and $v_k$ is measurement noise vector at time step $k$ such that $v_k = \mathcal{N}(0, c_2)$ or Gaussian white noise. $H$, which can also be expressed as a matrix since the system is linear, enables the state variables to be linearly transformed to match the outputs of the system. The dimensions of $H$ reflects which state variables have measurable values in the system. It is not assumed that every state variable is measurable, so $H$ allows us to compare the measurable state variables to $x_k$. Simple applications of $H$ include creating matrices with 0's and 1's, with 1's denoting that a state variable is measurable and a 0's representing non-measurable states. In other cases, $H$ is an integer used as a scaling factor. \\ 





\comment{
Before delving into the formal steps of the KF it is important to understand the underlying foundation of this algorithm. The goal of the Kalman Filter is to predict the new state of a system, after it undergoes some transformation. Let this be represented by 
\begin{align*}
	\frac{dx}{dt} = f(x) + \varepsilon ',
\end{align*}
where $\frac{dx}{dt}$ is the prediction of the states, $f$ is a linear function that transforms the states, given by $x$, and $ \varepsilon '$ is internal system noise. Since the system we are considering is linear, we can represent this transformation using matrix multiplication. Let matrix $M'$ be used to represent the transformation $f$ in order to approximate the equation above as
\begin{align*}
	\frac{dx}{dt} \approx M'x + \varepsilon '.
\end{align*}
    
\noindent The Kalman Filter works by generating predictions and making correction at various time steps. Let these time steps be denoted as $k$, such that $ k $ is a nonnegative integer and let $x_k$ be the estimate of the state variables at time $k$ with initial assumptions of the state variables beginning at $x_0$, such that $x_0 = \mathcal{N}(m_0, c_0)$, where $m_0$ and $c_0$ is mean and variance, respectively. 

\begin{figure}[h]
    \centering
    \includegraphics[scale = 0.3]{kgraph.png}
    \caption{A diagram showing discretized time steps of the system. Though this diagram depicts time steps that are equal, this should not be assumed for all cases.}
\end{figure}

\noindent Next, the system can be discretized by approximating $\frac{dx}{dt}$ using the limit definition of derivative:
\begin{align*}
	\frac{x_{k+1} - x_k}{\Delta t} &\approx M'x_k + \varepsilon '_k \\
	x_{k+1} - x_k &\approx M'x_k \Delta t + \varepsilon '_k  \Delta t \\
	x_{k+1} &\approx M'x_k \Delta t + \varepsilon '_k  \Delta t + x_k \\
	x_{k+1} &\approx (M' \Delta t + I)x_k + \varepsilon '_k  \Delta t.
\end{align*}


\noindent Next, substitute $M$ for $M' \Delta t + I$ and $\varepsilon_k$ for  $\varepsilon '_k  \Delta t $ to get the dynamics model:
\begin{align*}
	x_{k+1} \approx M x_k + \varepsilon_k,
\end{align*}
\noindent where $\varepsilon_k = \mathcal{N}(0, c_1)$ or Gaussian white noise. This general equation is significant to the generation of predictions and will continue to be used in different versions of the Kalman Filter. \\ 

}





\noindent Given the system model and observation model, the algorithm of the KF can be discussed in greater depth. As a general overview, the KF algorithm consists of three major components:
\begin{enumerate}
  \item Initialize state variables
  \item Generate a prediction
  \item Update prediction with measurements from the system.
\end{enumerate}
The recursive component of the filter consists of repeating steps 2 and 3 repeatedly, while step 1 only needs to be done once. Details about each step are explored further below. \\

\begin{enumerate}
  \item Begin by initializing the state estimate and the initial state covariance matrix. The state estimate, $x_0$,  is a  column vector containing state variables, call them $x_a, x_b, \hdots, x_n$, such that $x_0= [x_a, x_b, \hdots, x_n]^T$, where $T$ is the transpose. The state estimate can be found by taking the expected value of the data, which is normally distributed. If the system states are finite, the expectation is denoted by
    \begin{align*}
        \mathbb{E}[x_0]   = \sum^n_{i = a} x_i p_i = [x_a p_a + x_b p_b + \hdots + x_n p_n]^T,
        %= \begin{bmatrix}
          % x_a \\
           %\vdots \\
           %x_ 
        % \end{bmatrix}.
    \end{align*}
    and if the system states are continuous case, the expectation is denoted by 
    \begin{align*}
        \mathbb{E}[x_0]   = \int^n_{a} xf(x)  dx
    \end{align*}
    
    
    where $f(x)$ is the probability density function. The state covariance matrix, $P_0$, is a square matrix whose contents are the covariance of the pairwise elements
    \footnote{Recall that the covariance of a variable with itself is the variance of the variable}.  A state covariance matrix is a symmetrical positive semi-definite square matrix whose diagonals correspond to the variance of a variable at location $i$ and elsewhere is the covariance of the pairwise elements. In practice, covariance matrices help us better understand the spread of data. For the case of the KF, calculating the state covariance is necessary for computing Kalman Gain, which is used in the correction step. Calculating the state covariance matrix can be done by 
    \begin{align*}
      P_0 =
      \begin{bmatrix}
           var(x_a)  & \hdots & cov(x_a,x_n) \\
           \vdots & \ddots & \vdots \\
           cov(x_n, x_a)  & \hdots & var(x_n )
         \end{bmatrix} .
  \end{align*}
  Enough should be known about the modeled system to generate these values. In the case where one knows the true value of the initial states, the state covariance would have all 0 values. On the other hand, if one is unsure of the initial values, values in $P_0$ are expected to be higher. It is important to initialize the model with values close to the true value, else the system will converge at a slow rate.  \\ 
  
 \item After initializing the state estimate and state covariance, a prediction can be generated. The estimate of the system at the next time step, $x_{k+1}$,  is given by
  \begin{align*}
      x_{k|k-1} = F x_{k-1|k-1} + w_{k-1} ,
  \end{align*} 
  where $F$ is the state transition matrix and $w_k$ is the process noise vector. Every state variable contained in $x_k$ is defined by a linear differential equation. These linear differential equations can be used to generate the F matrix. Therefore, F should be a square matrix whose dimension is equal to the number of states variables. \\ \\
  $w_k$ is the process noise vector at time k. Process noise can be thought of as the model's accuracy. When process noise is 0, it implies that the model is perfectly accurate and does not have to correct for incoming system measurements. On the other hand, high process noise will essentially restart the system based on incoming measurements. $w_k$ has the same dimensions as $x_k$, allowing us to identify whether or not to adjust the equations for the state variables. \\
  
  \item Next, correct the prediction with incoming measurements from the system. Begin by calculating the state covariance matrix in order to calculate Kalman Gain. The state covariance matrix at time step $k$ given the last time step, is 
    \begin{align*} 
        P_{k | k -1} = F P_{k - 1} F^T + Q_{k-1}, 
    \end{align*}
    where $F^T$ is the transpose of $F$, and $Q_{k}$ is the process noise covariance of $w_k$.
    The Kalman Gain at time step $k$, is given by
    \begin{align*} 
        K_k = P_{k | k - 1} H^T (H P_{k | k - 1} H^T + R_k)^{-1},
    \end{align*}
      where $H^T$ is the transpose of the observation matrix and $R$ is the measurement noise covariance matrix. \\ 

     \noindent From this equation, one can see that balancing $Q$ and $R$ is critical for model performance. Larger values of $Q$ indicate higher modeling error, which leads to a higher Kalman Gain and increased model correction. On the other hand, large values of $R$ imply high measurement error, leading to a lower Kalman Gain and less model correction. \\ 
     
     \noindent The Kalman Gain is a measure of how much to change the model based on incoming data. Low values of the Kalman Gain imply the model is accurate while higher values indicate the model should adjust based on the incoming data.  \\ 
   
     Next, calculate the transformed prediction in order to correct the prediction. The transformed state vector, $\hat y_k$, is given by
    \begin{align*}
        \hat y_k = H x_{k|k-1} + v_k,
    \end{align*}
    
    where $v_k$ is measurement noise, which is added to account for measurement error.
    The corrected prediction, $ x_{k|k-1} $, is given by
    \begin{align*} 
        x_{k|k} = x_{k|k - 1} + K_k(y_k - \hat y_{k}),
    \end{align*}
   where  $y_k$ is the actual measurements of the system, $K_k$ is the Kalman Gain matrix at time step $k$, and $x_{k-1}$ are the values of the state variable at the last time step. The quantity $y_k - \hat y_{k}$ is also known as measurement residual or innovation.  Later on, we will see how this value is used to gauge model performance. \\
   
      The final step is to update the state covariance matrix, $P_k $, through the equation
    \begin{align*} 
        P_{k|k} = (I - K_k H) P_{k | k-1},
    \end{align*}
    where $I$ is the identity matrix, $K_k$ is the Kalman Gain at time step $k$, $H$ is the observation matrix, and $P_{k|k-1}$ is the state covariance at time step $k$ given the last time step. $ P_k $ will be used in the next iteration of the filter.
\end{enumerate} 
\newpage

\noindent In all, the KF can be used to generate predictions for linear dynamical systems by initializing a model, generating a prediction, and correcting the prediction. Initializing the model requires an understanding of the state's initial values; confidence in the accuracy of the initial values is reflected in the state covariance. After initializing the model, a prediction can be made. Since the system is linear, state transformation can be done by representing the ODE of the state as a matrix and performing matrix multiplication. Finally, the prediction can be updated by incorporating incoming system measurements and calculating the Kalman Gain. This process can be done repeatedly as long as measurements are being inputted into the system. A summary of all of the variables used in the KF can be found in Table ~\ref{tab:KF}. Though the KF is effective in computing predictions for linear dynamical systems, not all systems in the real world are linear. There are several iterations of the KF that have been adapted for non linear dynamical systems.

\begin{center}
\begin{table}[h]
\centering
\caption{Description of all variables in the Kalman Filter} \label{tab:sometab}
\begin{tabular}{ |p{2cm}||p{5cm}|p{2cm}| }
    \hline
    \multicolumn{3}{|c|}{Variables in the Kalman Filter } \\ 
    \hline
    Variable & Description & Dimensions \\
    \hline
    $x$ & State variables  & $x \times 1$ \\
    $\hat y$ & Transformed state vector  & $y \times 1$ \\
    $y$ & Actual system measurement(s) & $y \times 1$ \\
    $v$ & Measurement noise vector & $y \times 1$\\
    $w$ & Process noise vector & $x \times 1$\\
    $F$ & State function  & $x \times x $  \\ 
    $H$ & Observation function & $y \times x$\\
    $K$ & Kalman Gain  & $x \times y$\\
    $Q$ & Process noise covariance  & $x \times x$ \\
    $R$ & Measurement noise covariance &  $y \times y$\\
    $P$ & Covariance matrix & $x \times x $  \\ 
    \hline
\end{tabular}
\end{table}
\label{tab:KF}
\end{center} 





\chapter{Extended Kalman Filters}
\label{Extended Kalman Filters}

The Extended Kalman Filter (EKF) is the non-linear version of the Kalman Filter. Though it can be used for non linear equations, it is important to note that it is not an optimal estimator. For the most part, the EKF Algorithm is nearly identical to the KF algorithm. The critical difference is in linearizing the state and observation function. The EKF uses the Jacobean to linearly approximate the non-linear function around the mean of the Gaussian distribution. Skipping this step would result in the transformed data being non-Gaussian; though taking the Jacobean enables the transformation to remain Gaussian, it is not exact, resulting in some generalization. Linear approximation through a single point also makes the EKF inefficient when dealing with complex, higher order systems. Because of this, the model is highly subject to error, which can be somewhat reduced by setting accurate initial values. Though these flaws exist, the EKF performs strongly with applications of real time spatial fields, including navigation and positioning systems.  \\ \\

\section{Extended Kalman Filter Algorithm}

\begin{center}
    
\centering
\begin{tabular}{ |p{2cm}||p{5cm}|p{2cm}| }
    \hline
    \multicolumn{3}{|c|}{Variables in the Extended Kalman Filter } \\ 
    \hline
    Variable & Description & Dimensions \\
    \hline
   x & State variables  & $d_x \times 1$ \\
    y & Output vector  & $d_y \times 1$ \\
    u & System inputs  & $d_u \times 1$\\
    v & Measurement noise & $d_y \times 1$\\
    w & Process noise & $d_x \times 1$\\
    f & None linear state function  & $d_x \times d_x $  \\ 
    F & State function  & $d_x \times d_x $  \\ 
    h & Non linear observation function & $d_y \times d_x$\\
    H & Linearized observation function & $d_y \times d_x$\\
    K & Kalman Gain  & $d_x \times d_y$\\
    Q & Process noise covariance  & $d_x \times d_x$ \\
    R & Measurement noise covariance &  $d_y \times d_y$\\
    P & Covariance matrix & $d_x \times d_x $  \\ 
    \hline
\end{tabular}
\end{center}
\begin{enumerate}
  \item Initialize the state estimate ($x_0$) and the initial covariance matrix ($P_0$) 
  \begin{align*}
        \hat x_0 = \mathbb{E}[x_0]  = \begin{bmatrix}
           x_1 \\
           \vdots \\
           x_n 
         \end{bmatrix}  
    \end{align*}
  \begin{align*}
      P_0 =
      \begin{bmatrix}
           var(x_1)  & \hdots & cov(x_1, x_n) \\
           \vdots & \ddots & \vdots \\
           cov(x_n, x_1)  & \hdots & var(x_n )
         \end{bmatrix}  
  \end{align*}
  Often, $P_0 $ will be initialized as a diagonal matrix with the diaganals being the variance of each state variance and every other value being set to 0.
  \item Calculate the Jacobean
  \begin{align*}
      F= \frac{\partial f}{\partial x} =
      \begin{bmatrix}
           \frac{\partial f_1}{\partial x_1} & \hdots & \frac{\partial f_n}{\partial x_n} \\
           \vdots & \ddots & \vdots \\
           \frac{\partial f_n}{\partial x_1}  & \hdots & \frac{\partial f_n}{\partial x_n}
         \end{bmatrix}  
  \end{align*}
  \begin{align*}
      H = \frac{\partial h}{\partial x} =
     \begin{bmatrix}
           \frac{\partial h_1}{\partial x_1} & \hdots & \frac{\partial h_m}{\partial x_n} \\
           \vdots & \ddots & \vdots \\
           \frac{\partial h_m}{\partial x_1}  & \hdots & \frac{\partial h_{d_y}}{\partial x_n}
         \end{bmatrix}  
  \end{align*}
   Here, both $f$ and $h$ have no closed form solution. $f$ is the non-linear state function and $F$ is the linear approximation of $f$ with dimension $d_x$. Similarly, $h$ is the non linear observation while $H$ is linearized observation with dimension $d_y$. 
  \item Generate a prediction 
  \begin{align*}
      x_{k+1} = f( x_{k-1} , u_{k-1})  
  \end{align*} 
  This equation is different from the one used in the Kalman filter 
  \item Use the state covariance matrix ($P_{k | k - 1}$) to calculate Kalman Gain ($K_k$) 
    \begin{align*} 
        P_{k | k -1} = F P_{k - 1} F^T + Q_{k-1} 
        \end{align*}
         \begin{align*} 
        K_k = P_{k | k - 1} H^T_K (H_k P_{k | k - 1} H^T_K + R_k)^{-1}
    \end{align*}
    Recall that $H$ and $F$ are linear approximations of $f$ and $h$. Therefore, it is assumed that there is some amount of error when calculating the Kalman Gain.
    
    \item  Correct the prediction
    \begin{align*}
        \hat y_k = H( x_k + v_k )
    \end{align*}
     \begin{align*} 
        x_k = x_{k - 1} + K_k(y_k - \hat y_k)
    \end{align*}
    \item Update the covariance matrix
    \begin{align*} 
        P_k = (I - K_K H_k) P_{k | k-1}
    \end{align*}
\end{enumerate} 

\section{Modeling Metabolites using EKF}
\label{Modeling Metabolites using EKF}
\label{chap:EKF_Meskin}

\noindent This example is inspired by research that models the biological pathway of metabolites, which are molecules that are the byproduct of the body's metabolism. This model contains four different states, or metabolites, which contain 18 unknown parameters \cite{article5}. Unlike the original research, which had datasources of their own, this example will simulate data by using a built-in ODE solver on Matlab. The dataset used to generate these results are included along with the code in the appendix. Ultimately, the goals of this example is to demonstrate how the EKF can be applied, how the EKF can correct for multiple variables or states, and how the EKF can be used for parameter fitting. \\ 

\noindent In this example, the four metabolites, or states, have the following differential equations:
\begin{align*}
\dot x_1 &= \alpha_1 x_3^{g_{13}}  - \beta_1 x_1^{h_{11}}, \\
\dot x_2 &= \alpha_2 x_1^{g_{21}} - \beta_2 x_2^{h_{22}}, \\
\dot x_3 &= \alpha_3 x_2^{g_{32}} - \beta_3 x_3^{h_{33}} x_4^{h_{34}}, \\
\dot x_4 &= \alpha_4  x_1^{g_{41}} - \beta_4 x_4^{h_{44}},
\end{align*}
where there are 18 unknown parameters ($\alpha_1, \hdots, \alpha_4, \beta_1, \hdots, \beta_4, g_{13}, g_{15}, g_{21}, \\ g_{32}, g_{41}, h_{11}, h_{22}, h_{33}, h_{34}, h_{44} $) and the state vector is given by $x= [x_{1}, x_{2}, x_{3}, x_{4}]^T$. For now, we will use the true value of these parameters (which can be found in \cite{article5}), but later on the EKF will be applied for parameter estimation. In both the original example as well as this one, sampling time will be 0.1 seconds for 5 seconds, totaling 50 UKF estimates. \\



\clearpage
\subsubsection{Implementing EKF}

Recall that implementing the EKF requires initializing the model, generating a prediction, and correcting the prediction. Below outlines the process of how the EKF can be applied for a single time step. Applying the EKF to multiple time steps is simulated on Matlab, and the code for doing so can be found in the appendix.

\begin{enumerate}
\item The model is initialized by setting the state variable to $x_0 = [4, 1, 3, 4]^T$, which is very close to the true values of the system. Since the initialized state has values near the true value, the state covariance can also be set to a lower value of $P_0 = .01I$. 
\item The next step is to generate a prediction. To do so, the system must be discretized into time steps. The system is given in a continuous state space in the form of $\dot x = Ax$, and can be discretized by converting it into the form $\dot x = e^{At}x$. However, in this system, calculating the matrix exponential is computationally intense (taking days to run on a single iteration on Matlab). Therefore, the Euler method will be used with a time step of 0.05 seconds, resulting in
\begin{align*}
 x_{1|0} &=  x_{0|0} + 0.05  * f(x) \\ \\
&= \begin{bmatrix}
4 \\
1\\
3 \\
4
\end{bmatrix} + 0.05
\begin{bmatrix}
 \alpha_1 3^{g_{13}} - \beta_1 4 ^{h_{11}} \\
 \alpha_2 4^{g_{21}} - \beta_2 1^{h_{22}} \\
 \alpha_3 1^{g_{32}} - \beta_3 3^{h_{33}} 4^{h_{34}} \\
\alpha_4  4^{g_{41}} - \beta_4 4^{h_{44}}
\end{bmatrix} \\ \\
&=\begin{bmatrix}
3.4152 \\
1.6500 \\
2.5786 \\
3.2906
\end{bmatrix}.
\end{align*}

\item Finally, the prediction can be corrected. Calculate $F$, the jacobian of the nonlinear transformation $f=[\dot x_1, \dot x_2, \dot x_3, \dot x_4]^T$, and $H$, the Jacobian of the nonlinear observation function $h=[\dot x_1, \dot x_2, \dot x_3, \dot x_4]^T$, as follows
\begin{align*}
F = H = 
\begin{bmatrix}
-\beta_1 h_{11} x_1^{h_{11}-1} & 0 & x_5 x_3 g_{13} ^{g_{13}-1}& 0 \\
\alpha_2  g_{21} x_1^{g_{21}-1} & -\beta_2 h_{22} x_2^{h_{22}-1}&0&0\\
0&\alpha_3 g_{32} x_2^{g_{32}-1} & -\beta_3 h_{33} x_3^{h_{33}-1} x_4^{h_{34}}&-\beta_3 h_{34} x_4^{h_{34}-1} x_3^{h_{33}}\\
\alpha_4 g_{41} x_1^{g_{41}-1}&0&0&-\beta_4 h_{44} x_4^{h_{44}-1}
\end{bmatrix}.
\end{align*}




\noindent  $H$ should be adjusted according to how many states are being corrected. For instance, if correction is being done for all states, $H$ should remain as it is above. However, if only the first state is being corrected, then $h = [\dot x_1]$ and $H = [-\beta_1 h_{11} x_1^{h_{11}-1},  0,  x_3 g_{13} ^{g_{13}-1}, 0 ]$. \\

\noindent Assuming that the first state is the only state being corrected, $F$ and $H$ are used to calculate the Kalman Gain, which is 
\begin{align*}
K_1 = 
\begin{bmatrix}
.05402 & -.0370 & .0465 & .0149
\end{bmatrix}^T.
\end{align*}

\noindent Additionally, the prediction can be transformed in order to compare it with incoming system measurement. In the case where only the first state is being predicted, the prediction is transformed by applying the nonlinear observation function to the prediction. Similar to generating the prediction, this can be done by using Euler's method to discretize the system, resulting in
\begin{align*}
y_{1} &=  x_{1,1|0} + 0.05 * h(x_{1|0}) \\ 
&=3.4152  + 0.05(\alpha_1 2.5786 ^{g_{13}} - \beta_1 3.4152  ^{h_{11}} )=
2.9600.
\end{align*}

\noindent This transformed value is quite close to the measured value of 3.1126, meaning that in the first time step, the residual is approximately .15. Recall that the measured value was simulated on Matlab and can be found in the appendix.\\

\noindent Finally, the prediction can be corrected, resulting in
\begin{align*}
 x_{1|1}&=
  x_{1|0} + K_1 (y_1 - \hat y_1) =
\begin{bmatrix}
3.5751\\
1.5404 \\
2.7162 \\
3.3346
\end{bmatrix}.
\end{align*}

\end{enumerate}


\begin{figure}[ht]
    \centering
    \includegraphics[scale = 0.3]{EKF_1state.png}
    \caption{Implementing EKF with correction applied to State 1 and initialized with states $[4, 1, 3, 4]^T$.}
    \label{fig:EKF_1state}
	\vspace*{\floatsep}%
    \includegraphics[scale = 0.3]{EKF_4state.png}
    \caption{Using EKF with correction applied to all states using a model initialized with states $[4, 1, 3, 4]^T$.}
    \label{fig:EKF_4state}
\end{figure}



\noindent Applying the EKF for 50 time steps by repeating steps 2 and 3 on Matlab results in figure \ref{fig:EKF_1state}. Multiple state correction is similar to the implementation above. The only difference is in the correction step. Notice the improvement in performance when all states are being corrected, as seen in fig \ref{fig:EKF_4state}.  \\

\noindent For more examples on applying the EKF and MATLAB code, the following can be referenced \cite{cao_2008, article7}.






\clearpage
\subsubsection{EKF Parameter Estimation}
The EKF (as well as other versions of the Kalman Filter), can be utilized for parameter estimation. There are many methods of parameter estimation, which include using dual techniques. In fact, there is a nonlinear version of the Kalman Filter, called the Dual Unscented Kalman Filter, that was designed in order to simultaneously monitor state and parameter values. Though this thesis does not explore these methods, more information can be found in \cite{inbook, article6}. In particular, this thesis explores the use of joint parameter estimation. This technique involves creating new states for each of the unknown parameters. \\ 

\noindent In this example, the goal is to use the EKF to find parameter values that best fit this model to its true states. This example continues to use the metabolites system with the goal of estimating parameter $\alpha_1$. Begin by declaring $\alpha_1$ as a separate state. The new set of differential equations for this 5 state system is given by
\begin{align*}
\dot x_1 &= x_5 x_3^{g_{13}} - \beta_1 x_1^{h_{11}}, \\
\dot x_2 &= \alpha_2 x_1^{g_{21}} - \beta_2 x_2^{h_{22}}, \\
\dot x_3 &= \alpha_3 x_2^{g_{32}} - \beta_3 x_3^{h_{33}} x_4^{h_{34}}, \\
\dot x_4 &= \alpha_4  x_1^{g_{41}} - \beta_4 x_4^{h_{44}}\\
\dot x_5 &= 0.
\end{align*}


\noindent The ODE for $x_5$ is 0, because $x_5$ is a parameter, or constant, and does not change overtime. In this 5 state system, the nonlinear state transition function is equal to $f = [\dot x_1, \dot x_2, \dot x_3, \dot x_4, \dot x_5]^T$. By applying the EKF to this system while correcting for all 5 states, the EKF performs well, as shown in ~\ref{fig:EKF_1param}.

\begin{figure}[h]
    \centering
    \includegraphics[scale = 0.3]{EKF_1param.png}
    \caption{Results obtained by implementing the EKF with parameter estimation of $\alpha_1$. The model is initialized with $[4, 1, 3, 4, 20]^T$, which is close to the true initial value of the system. While not depicted in this visual, the model still performs well when initialized with values further from the true system values. }
    \label{fig:EKF_1param}
\end{figure}

\comment{
The jacobian of $f$, or $F$ is given by
$$
\scriptsize{
\begin{bmatrix}
-\beta_1 h_{11} x_1^{h_{11}-1} & 0 & x_5 x_3 g_{13} ^{g_{13}-1}& 0 & x_3^{g_{13}}\\
\alpha_2  g_{21} x_1^{g_{21}-1} & -\beta_2 h_{22} x_2^{h_{22}-1}&0&0&0 \\
0&\alpha_3 g_{32} x_2^{g_{32}-1} & -\beta_3 h_{33} x_3^{h_{33}-1} x_4^{h_{34}}&-\beta_3 h_{34} x_4^{h_{34}-1} x_3^{h_{33}}&0\\
\alpha_4 g_{41} x_1^{g_{41}-1}&0&0&-\beta_4 h_{44} x_4^{h_{44}-1}&0\\
0&0&0&0&0& \\
\end{bmatrix}}.
$$
}


\clearpage

\noindent Parameter estimation can be applied to more than one parameter. In this example, consider parameter estimation for four states, $\alpha_1, \alpha_2, \alpha_3, \alpha_4$. This introduces four new states into the original four state system, resulting in 
\begin{align*}
\dot x_1 &= x_5  x_3^{g_{13}} x_5^{g_{15}} - \beta_1 x_1^{h_{11}} , \\
\dot x_2 &= x_6  x_1^{g_{21}} - \beta_2 x_2^{h_{22}}, \\
\dot x_3 &= x_7  x_2^{g_{32}} - \beta_3 x_3^{h_{33}} x_4^{h_{34}}, \\
\dot x_4 &= x_8   x_1^{g_{41}} - \beta_4 x_4^{h_{44}} \\
\dot x_5 &= \dot x_6= \dot x_7 = \dot x_8 = 0.
\end{align*}

\comment{
F=
$$
\scriptsize{
\begin{bmatrix}
-\beta_1 h_{11} x_1^{h_{11}-1} & 0 & x_5 x_3 g_{13} ^{g_{13}-1}& 0 & x_3^{g_{13}} & 0 & 0 & 0\\
x_6  g_{21} x_1^{g_{21}-1} & -\beta_2 h_{22} x_2^{h_{22}-1}&0&0&0 & x_1^{g_{21}} & 0 &0\\
0&x_7 g_{32} x_2^{g_{32}-1} & -\beta_3 h_{33} x_3^{h_{33}-1} x_4^{h_{34}}&-\beta_3 h_{34} x_4^{h_{34}-1} x_3^{h_{33}}&0 & 0 &x_2^{g_{32}} &0 \\
x_8 g_{41} x_1^{g_{41}-1}&0&0&-\beta_4 h_{44} x_4^{h_{44}-1}&0 & 0 & 0 & x_1^{g_{41}}\\
0&0&0&0&0&0&0&0 \\
0&0&0&0&0&0&0&0 \\
0&0&0&0&0&0&0&0 \\
0&0&0&0&0&0&0&0 \\
\end{bmatrix}}.
$$s
}

\noindent The results of the EKF when applied to this 8 state system is shown in figure ~\ref{fig:4param}. 
After 50 time-steps, the EKF prediction of these four parameter is quite close to their true values, as shown in Table ~\ref{tab:4param}.

\begin{center}
\begin{table}[h]
\centering
\begin{tabular}{ |P{3cm}||P{1cm} P{3cm} P{3cm} |}
    \hline
    \multicolumn{3}{|c|}{Parameter Values} \\ 
    \hline
     Parameters & True & EKF estimate & IUKF Estimate\\
    \hline
    $\alpha_1$ & 20.0  & 20.0125 &19.9744 \\
    $\alpha_2$ & 8.0  & 7.9819  & 8.0017 \\
    $\alpha_3$ & 3.0  & 2.9818 & 3.0015 \\
    $\alpha_4$ & 2.0 & 1.9614  & 2.0016 \\
    \hline
\end{tabular}
\caption{This table shows the true values of the parameters, the final EKF prediction of the parameters, and the final IUKF prediction of the parameters. Here, the term final is being used to denote the performance of the filter after 50 time-steps.}
\label{tab:4param}
\end{table}
\end{center}

\begin{figure}[ht]
    \centering
    \includegraphics[scale = 0.5]{EKF_4param.png}
    \caption{EKF performance of an 8 state system with 4 parameters. This model was initialized with $[4;1;3;4;20;8;3;2]^T$, which is almost exactly the system's true values.}
    \label{fig:4param}
\end{figure}






\comment{
\noindent Residuals, also known as the innovation, are one way to access the model's performance. Recall that the residual is the difference between the actual measurement values and the predicted measurement values. Generally, a strong residual graph has

\begin{itemize}
\item a symmetrical distribution that is clustered toward the center,
\item values that are close to 0,
\item a random or unclear pattern.
\end{itemize}

\noindent Figure ~\ref{fig:4params} is the residual graph of the system where the EKF is applied to the four states and four parameters, $\alpha_1, \hdots, \alpha_4$. While this residual graph has values close to 0 and a symmetrical distribution, the pattern is similar to the behavior of the states. Therefore, though EKF can be applied to this nonlinear system, performance can be further optimized.

\begin{figure}[h]
    \centering
    \includegraphics[scale = 0.6]{EKF_4param_residual.png}
    \caption{Built in matlab, four state correction and 4 param}
    \label{fig:4params}
\end{figure}


\noindent There are many methods of parameter estimation, which include using dual techniques. In fact, there is a nonlinear version of the Kalman Filter, called the Dual Unscented Kalman Filter, that was designed in order to simultaneously monitor state and parameter values. Though this thesis does not explore these methods, more information can be found in \cite{inbook, article6}. 
}


















\chapter{Unscented Kalman Filters}
\label{Unscented Kalman Filters}

The Unscented Kalman Filter (UKF) is another nonlinear version of the Kalman Filter developed to address the shortcomings of the EKF. For instance, as opposed to using the Jacobean to linearly approximate around a single point, the UKF uses the Unscented Transform (UT) to approximate around multiple points, known as sigma points. The UT is a method of approximating probability distributions that have undergone a non linear transformation using limited statistics. The UT involves using these sigma points, which are represented in a Sigma Point Matrix, to represent the normal distribution of the data. The covariance and weights of these sigma points are calculated. These sigma points undergo a non-linear transformation, resulting in a posterior distribution that is not normal \cite{inbook, Wan01theunscented} . We are able to approximate the normal distribution of the posterior distribution using the weights and covariance that were calculated prior to the transformation. This process enables the Kalman Filter to be applied to more complex non linear problems. \\ \\
Unlike the Kalman Filter and the Extended Kalman Filter, the UKF also has a set of parameters. Explanations of each parameter and their default values can be found in the chart below. For the UKF, parameters are necessary for controlling the spread of sigma points. This was not needed for the EKF, since the EKF was only linearizing around the mean.
\textcolor{red}{expand here on how to tune parameters.} \\ \\
The term 'unscented' was arbitrarily coined by the developer of the UKF, Jeffrey Uhlmann. In an interview, he shares:
\begin{displayquote}
"Initially I only referred to it as the “new filter.” Needing a more specific name, people in my lab began referring to it as the “Uhlmann filter,” which obviously isn’t a name that I could use, so I had to come up with an official term. One evening everyone else in the lab was at the Royal Opera House, and as I was working I noticed someone’s deodorant on a desk. The word “unscented” caught my eye as the perfect technical term. At first people in the lab thought it was absurd—which is okay because absurdity is my guiding principle—and that it wouldn’t catch on. My claim was that people simply accept technical terms as technical terms: for example, does anyone think about why a tree is called a tree?"
\end{displayquote}

\section{Unscented Kalman Filter Algorithm}
\begin{center}
    
\centering
\begin{tabular}{ |p{2cm}||p{5cm}|p{2cm}| }
    \hline
    \multicolumn{3}{|c|}{Variables in the Unscented Kalman Filter } \\ 
    \hline
    Variable & Description & Dimensions \\
    \hline
    x & Vector containing state variables & $d_x \times 1 $\\ 
    $\chi $& Sigma Point Matrix &$ d_x \times (2 d_x + 1) $\\
    f & Nonlinear tate function & $d_x \times d_x $  \\ 
    h & Nonlinear observation function & $d_u \times d_x$\\
    u & Input vector  & $d_u \times 1$\\
    P & Covariance matrix & $d_x \times d_x $  \\
    Q & Process noise covariance & $d_x \times d_x $  \\
    R & Measurement noise covariance & $d_y \times d_y $  \\
     $W^{(m)}$ & Weight for state variable (mean) & scalar \\
    $W^{(c)}$ & Weight for covariance & scalar \\
    \hline
\end{tabular} 
\end{center}
\begin{center}
\begin{tabular}{ |p{1cm}||p{5cm}|p{2cm}| p{1cm}| }
    \hline
    \multicolumn{4}{|c|}{Parameters in the Unscented Kalman Filter } \\ 
    \hline
     & Description & Bounds & Default \\
    \hline
    $\alpha$ & Controls spread of sigma points & $0 < \alpha \leq 1$ & $.001$\\
    $\beta$ & Adjust sigma point weight & $\beta \geq 0$ & 2\\
    $\kappa $ & Sigma point weighting constant & $0 \leq \kappa \leq 3^{*}$  & 0 \\
    \hline
\end{tabular}
\end{center}
* Others use $\kappa  = 3 - d_x$ 

\begin{enumerate}
    \item Initialize state vector and covariance
    \begin{align*}
        \hat{x}_{0} &= \mathbb{E}[x_{0}] 
       \end{align*}
        \begin{align*}
        P_{x_{0}} &= \mathbb{E}[(x_{0}-\hat{x}_{0})(x_{0}-\hat{x}_{0})^{T}] 
    \end{align*}
    This step is the exact same as the first step of the KF and the EKF. We need to initialize these values so the model can begin generating and correcting predictions.
    
        \item Calculate sigma points \\ \\
        Sigma points characterize the distribution of the data. The number of sigma points is deterministic and depends on the dimensions of the system. In general, a UFK will have  $2 \cdot d_x$ + 1 sigma points, where $d_x$ represents the dimension of the state vector \cite{inbook, inproceedings, Wan01theunscented}.  We use the equation below to generate a scalar value that determines how spread out the sigma points are from the mean. 
         \begin{align*}
        \lambda = \alpha^{2}(d_{x}+\kappa)-d_{x} 
         \end{align*}
         $\alpha$ and $\kappa$ are both parameters that control for the spread of sigma points around the mean value of the state. The spread of the sigma points is proportional to $\alpha$ . For both $\alpha$ and $\kappa $,  the smaller the values are, the closer the sigma points are to the mean.\\ \\
       $\beta$ is a parameter that uses information regarding state distribution to adjust sigma points. $\beta$ has a default value of 2 if the data is Gaussian. 
    \begin{align*}
        \chi_{\ 0,k-1} &= \hat{x}_{k-1} 
     \end{align*}
             Since the goal is to characterize the distribution, set one of these sigma points to the mean. Half of the remaining points will be smaller than the mean and the other half will be larger than the mean.
     \begin{align*}
        \chi_{\ i,k-1} &= \hat{x}_{k-1} \pm \bigg(\sqrt{(d_{x}+\lambda )P_{x_{k-1}}}\bigg)_{i} \quad \quad \quad i=1,\dots,2 d_x + 1
        \end{align*}
        The square root of a matrix, call it $A$ satisfies the following condition: $A = B^2$. Note that $(\sqrt{(d_{x}+\lambda)P_{x_{k-1}}})$ is a matrix, and the $i$ subscript is the $i^{th}$ column of the matrix. Also note that $\chi_{\ i,k-1}$ is the $i^{th}$ column of the sigma point matrix at time $k-1$.
       

        \item Calculate the weights for each sigma point \\ \\
        Weights are scalars used to calculate posterior sigma points after they have undergone a nonlinear transformation. The weights are later used to approximate Gaussian mean and covariance. Weights can have positive or negative values, but will ultimately sum to 1 \cite{article6}. The subscript of the weight indicates which sigma point the weight is for. 
        \begin{align*}
        W^{(m)}_{0} = \frac{\lambda}{d_{x}+ \lambda} 
         \end{align*}
         This is the weight for the mean of the 0th sigma point. 
        \begin{align*}
        W^{(c)}_{0} = \frac{\lambda}{d_{x}+ \lambda} + (1 - \alpha^{2} + \beta) 
         \end{align*}
         This is the weight for the covariance of the 0th sigma point. 
        \begin{align*}
        W^{(m)}_{i} = W^{(c)}_{i} = \frac{\lambda}{2(d_{x}+ \lambda) } \quad \quad \quad i=1,\dots,2d_{x}
            \end{align*}
           The rest of the weights for the means and covariances of the other $2 d_x$ sigma points is calculated above.
      
           
        \item Generate a prediction
        \begin{align*}
        \chi_{k | k - 1} = f(\chi)
        \end{align*}
        The equation above performs nonlinear transformation $f$ on the sigma points. Though $\chi$ has a Gaussian distribution,  $ \chi_{k | k - 1} $ does not because it has been transformed by the nonlinear state function $f$.
        \begin{align*}
        \hat x_{k | k-1} = \sum^{2d_x}_{i = 0} W_i^{(m)} \chi_{i, k | k - 1}
        \end{align*}
        Now, the new estimate of the state variables is conditioned on the last estimate. The weights are included in this step in order to approximate the Gaussian distribution.
        \begin{align*}
        P_{x, k | k-1} = \sum^{2d_x}_{i = 0} W_i^{(c)} (\chi_{i, k | k - 1} -  \hat x_{i, k | k - 1} )(\chi_{i, k | k - 1} -  \hat x_{i, k | k - 1} )^T + Q
        \end{align*} 
        The posterior covariance matrix for the state variable is necessary for updating the state covariance later on (step 9). Recall that Q is process noise, which provides the error in our model $f$.
        \item Calculate posterior sigma points \\ \\
        By now, the prediction step has concluded and the model begins the process of correction. Calculation of the posterior (also called augmented) sigma points is necessary for interpreting the distribution. From the non linear transformation above, the outputs are not Gaussian. Using the weights calculated in step 3, we are able to approximate the Gaussian distribution of the transformed sigma points.
         \begin{align*}
        \chi^{(aug)}_{0, k|k-1} = \hat x_{k|k-1}
        \end{align*}
         \begin{align*}
        \chi^{(aug)}_{ i,k |k-1} &= \hat{x}_{k |k-1} \pm \bigg(\sqrt{(d_{x}+\lambda)P_{x_{k |k-1}}} \bigg)_{i} \quad \quad \quad  i=1,\dots,2d_{x}
        \end{align*}
        The calculation is the same process as in step 2. Recall that $\lambda$ was calculated in step 2. Since it is not time dependent, we can use the same $\lambda$ value used earlier.
        
        \item Calculate transformed output 
         \begin{align*}
       \mathcal{Y}_{k|k-1} = h(\chi^{(aug)}_{k|k-1}) 
       \end{align*}
       \begin{align*}
       y_{k|k-1} = \sum^{2d_x}_{i = 0} W_i^{(m)}  \mathcal{Y}_{i, k | k - 1}
       \end{align*}
       Similar to the KF and EKF, use observation function $h$ to convert the sigma point matrix into ta form that can be compared to the actual measurements of the system $y_k$. Since this form is not Gaussian, use weights to interpret the results.
       
       \item Calculate Kalman Gain \\ \\
       Unlike previous versions of the KF, in addition to calculating the covariance of the state variables, calculations is also done for the covariance of observations  $P_{y}$ and for state variables with observations $P_{xy}$. When generating the covariance matrix for y the covariance of measurement noise is added. 
        \begin{align*}
       P_{y, k | k-1} = \sum^{2d_x}_{i = 0} W_i^{(c)} (\mathcal{Y}_{i, k | k - 1} -   y_{i, k | k - 1} )(\mathcal{Y}_{i, k | k - 1} -  y_{i, k | k - 1} )^T + R
       \end{align*}
        \begin{align*}
       P_{xy, k | k-1} = \sum^{2d_x}_{i = 0} W_i^{(c)} (\chi^{(aug)}_{i, k | k - 1} -  \hat x_{i, k | k - 1} )(\mathcal{Y}_{i, k | k - 1} -  y_{i, k | k - 1} )^T 
       \end{align*}
       \begin{align*}
       K_k = P_{xy, k | k-1} (P_{y, k | k-1}) ^{-1}
       \end{align*}
        
      \item Correct the prediction      
      \begin{align*}
       \hat x_{k} = \hat x_{k|k-1} + K_k(y_k - y_{k|k-1})
        \end{align*}
        Similar to the KF and EKF, the correction step follows the same equation. 
      
      
      \item Update the covariance matrix 
       \begin{align*}
       P_{x, k} = P_{x, k|k-1} -K_k (P_{y, k | k-1} ) {K_k}^T     
       \end{align*}     
            
            
            
            
            
            
            
            
            
            
            
            
            
            
            
            
            
            
    
    
        
\end{enumerate}
\section{Van Der Pol Example}
\label{Van Der Pol Example}
\section{Modeling Metabolites using UKF}
\label{Modeling Metabolites using UKF}




\noindent This is the same system explored in chapter~\ref{chap:EKF_Meskin}, but explored in the context of using an UKF as opposed to an EKF. The original research used an adaptation of the UKF, called the Iterative Unscented Kalman Filter (IUKF), to model the biological pathway of metabolites. Recall that this model contains four different states and 18 unknown parameters. These researchers utilized the IUKF for parameter fitting and was useful in enabling the model converge faster by resetting the covariance to re-excite the model. By not resetting the covariance at this step (as is done in the UKF), the three state variables without measurements converge significantly slower in this model. Utilizing the IUKF on this model was effective, because data regarding metabolites is highly influenced by noise, which is a factor that makes other approaches, such as regression and annealing, fail \cite{article5}. \\ 

\noindent In the paper, researchers had access to their own data sources and used an approach that was adapted from the UKF. Though we are not using their exact dataset, we will be simulating data using the same approach as the previous example. However, this example will be following UKF algorithm, as opposed to the IUKF algorithm. By doing so, the state variables without incoming measurements converge significantly slower than the measurable states. Ultimately, the goal of this example is to demonstrate how the UKF works on higher dimensional and more complex systems and how the UKF can be utilized in parameter estimation. \\

\noindent Recall that the four metabolites have the following differential equations:
\begin{align*}
\dot x_1 &= \alpha_1 x_3^{g_{13}} x_5^{g_{15}} - \beta_1 x_1^{h_{11}}, \\
\dot x_2 &= \alpha_2 x_1^{g_{21}} - \beta_2 x_2^{h_{22}}, \\
\dot x_3 &= \alpha_3 x_2^{g_{32}} - \beta_3 x_3^{h_{33}} x_4^{h_{34}}, \\
\dot x_4 &= \alpha_4  x_1^{g_{41}} - \beta_4 x_4^{h_{44}},
\end{align*}
with 18 parameters ($\alpha_1, \hdots, \alpha_4, \beta_1, \hdots, \beta_4, g_{13}, g_{15}, g_{21}, g_{32}, g_{41}, h_{11}, h_{22}, h_{33}, h_{34},\\ h_{44} $). In both the original example as well as this one, sampling time will be 0.1 seconds for 5 seconds, totaling 50 UKF estimates. Recall that data is simulated on MATLAB and the model is initialized with state variable $x_0 = [4, 1, 3, 4]^T$ and the state covariance to $P_0 = .01I$. \\

\noindent In the original example, researchers used the following hyper-parameter values: $\epsilon = 1, \kappa = -14$. In theory, values of $\kappa$ can be negative, but negative hyper-parameter values cannot be inputted into MATLAB. For this example, the hyper-parameter values used in this example are set to Matlab default values ($\alpha = 1e-3, \beta = 2, \kappa = 0$). The performance of the UKF on this system with these default hyper parameter values is shown in Figure ~\ref{fig:UKF_states}. Future work includes looking into ways to tune this hyper-parameters. \\

\comment{
\noindent Since the first state, $x1$ is the only state that has incoming system measurements, we have a 3 different blue lines in this figure. From this figure, it appears that the UKF prediction perfectly overlaps with the measured values. However, upon closer inspection, it actually does not. This is likely because there is only a small amount of measurement noise to the system. Figure 4.9 highlights the same information as Figure 4.8, but separates the states into different graphs, enabling a clearer illustration of each state's behavior.
}

\begin{figure}[ht]
    \centering
    \includegraphics[scale = 0.3]{UKF_states.png}
    \caption{Performance of all four state variables with all four states being corrected. The model is initialized with $x_0 = [4, 1, 3, 4]^T$, which is close to the system's true values and is why the system quickly converges with the true values. The measured and true values are very close together because the system has low values of measurement noise, $R=0.01$, and process noise, $Q=diag([0.2 0.1 0.3 .4 .2 .3 .2 .1])$. The values for $Q$ and $R$ were not taken from the original example \cite{article5}, which is a factor to consider when comparing results.}
    \label{fig:UKF_states}
\end{figure}

\comment{
\noindent An attempt was made to observe how the system changes with differing values of noise. In the real world, measurement noise values of $R=0.01$ is quite low; therefore, there was interest in seeing how increasing this value would change the system's behavior. However, }


\clearpage
\subsubsection{Parameter estimation}

Similar to chapter chapter~\ref{chap:EKF_Meskin}, joint parameter estimation can also be applied to the UKF. This example will be using the exact same dataset for measured and true system values as the one used in the EKF version. In order to estimate parameters, $\alpha_1,\alpha_2, \alpha_3, \alpha_4$, declare them as new states. Recall from the earlier EKF example that the system is as follows
\begin{align*}
\dot x_1 &= x_5  x_3^{g_{13}} x_5^{g_{15}} - \beta_1 x_1^{h_{11}} , \\
\dot x_2 &= x_6  x_1^{g_{21}} - \beta_2 x_2^{h_{22}}, \\
\dot x_3 &= x_7  x_2^{g_{32}} - \beta_3 x_3^{h_{33}} x_4^{h_{34}}, \\
\dot x_4 &= x_8   x_1^{g_{41}} - \beta_4 x_4^{h_{44}} \\
\dot x_5 &= \dot x_6= \dot x_7 = \dot x_8 = 0.
\end{align*}

\noindent The system is initialized with the same values as the EKF four parameter example, with the initial state being $x_0 = [4, 1, 3, 4, 20, 8, 3, 2]^T$ and the state covariance being $P_0 = .01I$. Also, the values of $Q$ and $R$ also remain the same. The results of the UKF on this 8 state system is illustrated in ~\ref{fig:UKF_4param}. Compared with the EKF results, the UKF seems to do poorer for State 3. However, in terms of parameter estimation, the EKF and UKF version seem to have the exact same performance, as shown in ~\ref{tab:UKF_4param}. These results may possibly be adjusted by changing hyper parameter values and can be explored in future work.

\begin{figure}[h]
    \centering
    \includegraphics[scale = 0.5]{UKF_4param.png}
    \caption{ds}
    \label{fig:UKF_4param}
\end{figure}

\begin{center}
\begin{table}[h]
\centering
\begin{tabular}{ |P{2cm}||P{1cm} P{2.5cm} P{2.5cm} P{2.5cm} |}
    \hline
    \multicolumn{5}{|c|}{Parameter Values} \\ 
    \hline
     Parameters & True & EKF estimate & UKF estimate &IUKF estimate \\
    \hline
    $\alpha_1$ & 20.0  & 20.0125 &20.0125&19.9744 \\
    $\alpha_2$ & 8.0  & 7.9819  &7.9819& 8.0017 \\
    $\alpha_3$ & 3.0  & 2.9818 &2.9818& 3.0015 \\
    $\alpha_4$ & 2.0 & 1.9614 &1.9614 & 2.0016 \\
    \hline
\end{tabular}
\caption{This table shows the true values of the parameters, the final EKF prediction of the parameters (as a reference to compare filter performance), the final UKF prediction of parameters, and the final IUKF prediction of the parameters. Here, the term final is being used to denote the performance of the filter after 50 time-steps. Interestingly, the performance of the EKF is the exact same as the performance of the UKF. By changing the hyper-parameters for the UKF, the model may be able to provide more accurate results.}
\label{tab:UKF_4param}
\end{table}
\end{center}

\clearpage

\clearpage
\noindent Residuals, also known as the innovation, are one way to access the model's performance. Recall that the residual is the difference between the actual measurement values and the predicted measurement values. Generally, a strong residual graph has

\begin{itemize}
\item a symmetrical distribution that is clustered toward the center,
\item values that are close to 0,
\item a random or unclear pattern.
\end{itemize}

\noindent Figure ~\ref{fig:4params} is the residual graph of the system where the EKF is applied to the four states and four parameters, $\alpha_1, \hdots, \alpha_4$. While this residual graph has values close to 0 and a symmetrical distribution, the pattern is similar to the behavior of the states. Therefore, though UKF can be applied to this nonlinear system, performance can be further optimized.

\begin{figure}[h]
    \centering
    \includegraphics[scale = 0.6]{UKF_4param_residual.png}
    \caption{Built in matlab, four state correction and 4 param}
    \label{fig:4params}
\end{figure}

\noindent Future work includes using the UKF for parameter fitting. One approach would be to use a technique similar to the one employed in chapter 3.2. Alternatively, another non-linear Kalman Filter technique, known as the Dual Unscented Kalman Filter, may also be a promising solution. Either of these two methods will likely be better than the method used in chapter 3.2, because calculating the exponential matrix and jacobian of the system is very computationally expensive and can be avoided with the UKF.

%\noindent \textcolor{red}{Expand here on correcting for 1+ states, currently we are having technical challenges in completing this step}















\chapter{Future Work}
\label{Future Work}


This thesis was inspired by previous work that involved modeling a system that forecasted glucose in real time in patients with Type 2 Diabetes. Researchers utilized the Dual Unscented Kalman Filter for simultaneous state and parameter estimation \cite{article1}. The goal of this thesis was to create a similar model for Type 1 Diabetes (T1D). In previous work by \cite{Shtylla}, a 12 state mathematical model for T1D was developed. The relationship between these 12 states are shown in Figure ~\ref{fig:relations}. This visualization demonstrates that this system is highly complex, as demonstrated by the amount of interconnectedness between the states. In addition, this system has more dimensions than the examples considered in this paper.

\begin{figure}[h]
    \centering
    \includegraphics[scale = 0.37]{t1d_model.png}
    \caption{This image is taken from \cite{Shtylla} and provides a visualization on the relationships between the 12 states in this Type 1 Diabetes system.}
    \label{fig:relations}
\end{figure}

The 12 state system is given as 
\begin{align*}
    &\frac{d}{dt} G = R_0 - (G_0 + S_I I)G, \\ 
    &\frac{d}{dt}  I = \sigma_1 K_1 (G, G_1) B - \delta_1 I, \\
    &\frac{d}{dt} M = J +(k+b)M_a -cM - f_M MB_a - f_M MB_n - e_1M(M+M_a), \\
    & \frac{d}{dt} M_a = f_M MB_a +f_M MB_n - kM_a - e_2M_a(M + M_a),\\
    &\frac{d}{dt} B = \alpha_B K_1(G)B - \delta_B B - \eta_e(t)K_2(E,R)B - W(B, t),\\
    & \frac{d}{dt} B_a = \Tilde{\delta_B} + \Tilde{\eta_e} (t)K_2(E,R)B + \Tilde{W}(B, t), \\
    & \quad \quad  \quad \quad - dB_a - f_M MB_a - f_{M_a} B_a - f^{tD}(D_{SS} - D)B_a - f_D DB_a, \\
    &\frac{d}{dt} B_n = dB_a - f_M MB_n - f_{M_a} M_a B_n - F_{Td}(D_{SS} - D)B_n - f_D DB_n, \\
    & \frac{d}{dt} D = f_{tD}B_n(D_{SS} - D - tD) + f_{tD}B_n tD - b_{DE} ED - \mu_D D, \\
    & \frac{d}{dt} tD = f_{tD}B_a(D_{SS} - D - tD) - f_{tD}B_n tD - b_{IR}RtD - \mu_D tD, \\
    & \frac{d}{dt} E = a_E (T_{naive} - E) + b_p \frac{DE}{\theta_D + D} - r_{am}E + b_EDE_m - \mu_E ER, \\
    & \frac{d}{dt} R = a_R (T_{naive} - R) + b_p \frac{tDR}{\theta_D + tD} - r_{am}R + b_R tDEm - \mu_R ER, \\
    & \frac{d}{dt} Em = r_{am}(E + R) - (a_{Em} + b_E D + b_R tD)eM.
\end{align*}

\noindent Future direction includes utilizing parameter estimation techniques from this thesis and applying them to this system. Since this system has several more dimensions and is more complex than other systems considered in this thesis, results using the UKF will likely be better than that of the EKF. In addition, further study of alternative parameter estimation techniques, especially the Dual Unscented Kalman Filter, can be beneficial. 



\bibliographystyle{plain}
\bibliography{bib.bib}




\appendix
\chapter{Van Der Pol Code}
\label{Van Der Pol Code}

\chapter{Meskin Code}
\label{Meskin Code}

\chapter{Albers Code}
\label{Albers Code}


\end{document}
