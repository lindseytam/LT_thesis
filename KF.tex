\chapter{Kalman Filters}
\label{Kalman Filters}

The Kalman Filter (KF) generates predictions for linear dynamical systems \cite{inbook}. Assuming the given data is noisy and Gaussian \cite{inproceedings, article7}, the first predictive step assumes knowledge of initial states and the model process, which takes the form of matrix $F$. The next step involves calculating the covariance in order to calculate the Kalman Gain, which is a measure of how much the estimate should be changed given actual measurements of the system. The corrective step utilizes the Kalman Gain and the measured system outputs to gauge whether to depend more on the model or on the data.  \\ \\
This process can be done recursively, allowing the model to become progressively more accurate as more data is added. Therefore, overtime, it is expected that the model will converge with the actual system measurements.


\section{Kalman Filter Algorithm}
\begin{center}
\centering
\begin{tabular}{ |p{2cm}||p{5cm}|p{2cm}| }
    \hline
    \multicolumn{3}{|c|}{Variables in the Kalman Filter } \\ 
    \hline
    Variable & Description & Dimensions \\
    \hline
    x & State variables  & $d_x \times 1$ \\
    y & Output Vector  & $d_y \times 1$ \\
    u & System Inputs  & $d_u \times 1$\\
    v & Measurement Noise & $d_y \times 1$\\
    w & Process Noise & $d_x \times 1$\\
    F & State function  & $d_x \times d_x $  \\ 
    H & Observation function & $d_y \times d_x$\\
    G & Input function & $d_x \times d_u$\\
    K & Kalman Gain  & $d_x \times d_y$\\
    Q & Process noise covariance  & $d_x \times d_x$ \\
    R & Measurement noise covariance &  $d_y \times d_y$\\
    P & Covariance matrix & $d_x \times d_x $  \\ 
    
    \hline
\end{tabular}
\end{center} 

\begin{enumerate}
  \item Initialize the state estimate ($x_0$) and the initial state covariance matrix ($P_0$) 
    \begin{align*}
        x_0 = \mathbb{E}[x_0]  = \begin{bmatrix}
           x_1 \\
           \vdots \\
           x_n 
         \end{bmatrix} 
    \end{align*}
  \begin{align*}
      P_0 =
      \begin{bmatrix}
           var(x_1)  & \hdots & cov(x_1,x_n) \\
           \vdots & \ddots & \vdots \\
           cov(x_n, x_1)  & \hdots & var(x_n )
         \end{bmatrix} 
  \end{align*}
  Enough should be known about the modeled system to generate these values. In general, $x_k$ denotes the model's estimate of the state variables, but when initializing these values, it is acceptable to use actual measurements of the system. \\ \\
   A state covariance matrix is a symmetrical positive semi-definite square matrix whose diagonals correspond to the variance of a variable at location i and elsewhere is the covariance of the pairwise elements. In practice, covariance matrices help us better understand the spread of data. For the case of the KF, calculating the state covariance is necessary for computing Kalman Gain. 
  \item Generate a prediction ($x_{k+1}$)
  \begin{align*}
      x_{k+1} = F x_{k-1} +  G u_{k-1} + w_{k-1} 
  \end{align*} 
  State matrix F and Input matrix G is how the system is defined. Every state variable contained in $x_k$ is defined by a linear differential equation. These linear differential equations can be used to generate the F and G matrices. Therefore, F should be a square matrix whose dimension is equal to the number of states variables and G can be a matrix (or in some cases a vector), depending on the dimension of input vector u. \\ \\
  $u_k$  is an input vector, which is a measurable value that helps define the system, but is not contained in the state vector. Depending on how the system is defined, $u_k$  can be a constant value or it can be a value dependent on time step k. \\ \\
  $w_k$ is the process noise vector at time k. Process noise can be thought of as the model's accuracy. When process noise is 0, it implies that the model is perfectly accurate and does not have to correct for incoming system measurements. On the other hand, high process noise will essentially restate the system based on incoming measurements. $w_k$ has the same dimensions as $x_k$, allowing us to identify whether or not to adjust the equations for the state variables. When all values of  $w_k$ equal 0, it implies that the linear differential equations we are using to define the state variables have no error. 
  \item Use the state covariance matrix ($P_{k | k - 1}$) to calculate Kalman Gain ($K_k$) 
    \begin{align*} 
        P_{k | k -1} = F P_{k - 1} F^T + Q_{k-1} 
    \end{align*}
    \begin{align*} 
        K_k = P_{k | k - 1} H^T_K (H_k P_{k | k - 1} H^T_K + R_k)^{-1}
    \end{align*}
      $H$ is the observation matrix, which enables the state variables to be linearly transformed to match the outputs of the system. The dimensions of $H$ reflects which state variables have measurable values in the system. It is not assumed that every state variable is measurable, so $H$ allows us to compare the measurable state variables to $x_k$. Simple applications of $H$ include creating matrices with 0's and 1's, with 1's denoting that a state variable is measurable and a 0's representing non-measurable states. In other cases, $H$ is an integer used as a scaling factor. \\ \\
     The Kalman Gain is the main component of the corrective aspect of the KF. The Kalman Gain is a measure of how much to change the model based on incoming data. Low values of the Kalman Gain imply the model is accurate while higher values indicate the model should adjust based on the incoming data. Calculating Kalman Gain involves using $Q$, which is process noise covariance of $w_k$,   $R$, which is measurement noise covariance, and $H$, which is also used in transforming the output vector in a later step. 
     From this equation, one can see that balancing $Q$ and $R$ is critical for model performance. Larger values of $Q$ indicate higher modeling error, which leads to a higher Kalman Gain and increased model correction. On the other hand, large values of $R$ imply high measurement error, leading to a lower Kalman Gain and less model correction. 
    \item Calculate the transformed output vector ($\hat y_k$) and correct the prediction  
    \begin{align*}
        \hat y_k = H_k x_k + v_k
    \end{align*}
    \begin{align*} 
        x_k = x_{k - 1} + K_k(y_k - \hat y_{k})
    \end{align*}
        $y_k$, also known as the output vector, is the actual measurements of the system and $\hat y_k$ is a way of transforming the prediction (using $H$) into a format that can be compared with $y_k$.The term $v_k$, also known as measurement noise, is also added to account for measurement error. Subtracting the actual measurements from the predicted measurements is necessary, because $y_k$ does not always contain measurements for all variables in state vector $x_K$. The quantity $y_k - \hat y_{k}$ is also known as measurement residual or innovation.  \\
     \item Update the covariance matrix
    \begin{align*} 
        P_k = (I - K_K H_k) P_{k | k-1}
    \end{align*}
    $ P_k $ will be used in the next iteration of the filter.
\end{enumerate} 




