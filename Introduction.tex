\chapter{Introduction}
\label{Introduction}

This paper explores the Kalman Filter, Extended Kalman Filter, and the Unscented Kalman Filter. The Kalman Filter is a data prediction method that estimates the value of states in linear dynamical systems. This is a recursive predictive-corrective process that enables the Kalman Filters to continuously generate predictions about state variables without relying on large amounts of initial data.  In the case of nonlinear systems, alternative forms of the Kalman Filter were developed, including the Extended Kalman Filter (EKF) and the Unscented Kalman Filter (UKF). The EKF linearizes the nonlinear system around the mean by taking the Jacobean of the nonlinear function. Navigation and signal processing are the main applications of the EKF Instead of linearizing the system around a single point, the UKF utilizes many points, known as Sigma points. The Sigma points then undergo a nonlinear transformation known as the Unscented Transform. Applications of the UKF includes modeling biological systems. Though we will discuss all three versions, the exploration of the UKF will be the main focus of this paper.  \\ \\
In addition to understanding the theory behind these algorithms, this paper also explores applications of the UKF. An example of the UKF can be applied to the Van der Pol oscillator, which is a self sustaining nonconservative oscillator. The purpose of this simple example is to demonstrate how a simple version of UKF can be implemented on MATLAB. We will also observe how process and measurement noise influences the system. A second example includes modeling the biological pathway of metabolites, which are molecules in the human body that are the byproduct of the metabolism. This example is slightly more complex, with four different states and 18 unknown parameters. The focus of this example is to illustrate the following three things: how the UKF can correct for multiple states, working with systems in higher dimensions, and how the UKF can be applied to extract parameter values. \\ \\
The ultimate goal of researching the UKF is to create a model that can forecast glucose values in real time of patients with Type 1 diabetes. This is inspired by a separate study that was able to do real-time glucose forecasting in Type-2 diabetes patients. Using an existing model of Type 1 diabetes that includes 12 states, the UKF will be most useful in extracting parameter values. To gauge the effectiveness of the model, we will be comparing the results with mouse data. Implementing this model has the potential to improve treatment methods and the quality of life of Type 1 diabetes patients.




