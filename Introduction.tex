\chapter{Introduction}
\label{Introduction}

This paper explores the Kalman Filter, Extended Kalman Filter, Unscented Kalman Filter (UKF), and the Dual Unscented Kalman Filter. The Kalman Filter is a data prediction method that estimates the value of unknown parameters of linear dynamical systems. This is a recursive predictive-corrective process that allows Kalman Filters to continuously generate predictions about state variables without relying on large amounts of initial data.  In the case of nonlinear systems, alternative forms of the Kalman Filter were developed, including the Extended Kalman Filter, the Unscented Kalman Filter, and the Dual Unscented Kalman Filter. The Extended Kalman filter linearizes the nonlinear system around the mean by taking the Jacobean of the nonlinear function. Navigation and signal processing are the main applications of the Extended Kalman Filter. Instead of linearizing the system around a single points, the Unscented Kalman Filter utilizes many points, known as Sigma points. The Sigma points then undergo a nonlinear transformation known as the Unscented Transform. The Dual Unscented Kalman Filter is similar to the Unscented Kalman Filter, but also estimates the value of some unknown parameter in the system. Applications of the Unscented and Dual Unscented Kalman Filters include modeling biological systems.  \\ \\
In addition to understanding the theory behind these algorithms, this paper also explores applications of the UKF. An example of the UKF can be applied to the Van der Pol oscillator, which is a self sustaining nonconservative oscillator.





