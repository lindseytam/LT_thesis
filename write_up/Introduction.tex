\chapter{Introduction}
\label{Introduction}

\noindent This thesis explores the theory and computational implementation of the Kalman Filter (KF), Extended Kalman Filter (EKF), and the Unscented Kalman Filter (UKF). The KF is a data assimilation method that estimates the value of states in linear dynamical systems. Beyond state estimation, the KF can also be used for parameter estimation. The method of parameter estimation explored in this thesis is known as joint parameter estimation and involves creating new states for the estimated parameters.\\ 

\noindent Chapter 2 introduces state space models and methods of discretization, which are important foundations for implementing the KF. The two methods of discretization include using the matrix exponential and Euler's method. Included in this chapter is a brief example that demonstrates how to put a system into its state space format. \\ 

\noindent Chapter 3 explores the theory and algorithm of the KF. The KF is a recursive predictive-corrective process that enables the continuous generation of predictions about state variables without relying on large amounts of initial data. Applications of the KF include navigation and mage processing. \\ 

\noindent In the case of nonlinear dynamical systems, alternative forms of the Kalman Filter were developed, including the EKF and the UKF. Chapter 4 explains the theory of the EKF and its ability to linearize the system around the mean by taking the Jacobian of the nonlinear function. Navigation and signal processing are the main applications of the EKF. This chapter also implements an example of the EKF in both state and parameter estimation using a biological system that models metabolites. \\

\noindent Chapter 5 discusses the theory of the UKF. Instead of linearizing the system around a single point, the UKF utilizes many points. Applications of the UKF includes parameterization of math models in biology. This chapter also implements two examples of the UKF: state estimation of the Van der Pol oscillator as well as state and parameter estimation in the same metabolites example from Chapter 4.  \\ 

\noindent The ultimate goal of researching the UKF is to create a model that can forecast glucose values in real time of patients with Type 1 diabetes. Chapter 6 discusses how using the UKF for parameter estimation can be applied to an existing 12 state model for Type 1 Diabetes. Implementing this model has the potential to improve treatment methods and the quality of life of Type 1 diabetes patients.


\comment{
\noindent In the case of nonlinear dynamical systems, alternative forms of the Kalman Filter were developed, including the EKF and the UKF. The EKF linearizes the nonlinear system around the mean by taking the Jacobian of the nonlinear function. Navigation and signal processing are the main applications of the EKF. Instead of linearizing the system around a single point, the UKF utilizes many points. Applications of the UKF includes parameterization of math models in biology. Beyond state estimation, these filters can also be used for parameter estimation. The method of parameter estimation explored in this thesis is known as joint parameter estimation and involves creating new states for the estimated parameters. \\ 

\noindent In addition to understanding the theory behind these algorithms, this thesis also explores how they can be applied to real world systems. The first example observes a system that models the pathway of human metabolites, which are the byproducts of the metabolism. The goal of this example is to understand how the EKF works, how it can be used for multiple state correction, and how parameter estimation can be applied. A second example explored in this paper involves applying the UKF to the Van der Pol oscillator, which is a self sustaining nonconservative oscillator. The purpose of this simple example is to demonstrate how a simple version of UKF can be implemented on MATLAB while also observing how process and measurement noise influences the system. The third and final example uses the same metabolites system as the first example, but applies it in the context of the UKF. The parameter estimation results from the EKF and UKF on the metabolites system, are both promising. Though performance of the two is similar, the UKF can be further optimized by adjusting the model's hyper-parameters, which the EKF does not have. \\ 

\noindent  The ultimate goal of researching the UKF is to create a model that can forecast glucose values in real time of patients with Type 1 diabetes. This is inspired by a separate study that was able to do real-time glucose forecasting in Type 2 diabetes patients. Using an existing model of Type 1 diabetes that includes 12 states, the UKF will be most useful in extracting parameter values. Implementing this model has the potential to improve treatment methods and the quality of life of Type 1 diabetes patients.
}




