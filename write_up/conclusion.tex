\chapter{Conclusion}
\label{Conclusion}

\noindent In all, this thesis explored the theory of the KF, EKF, and UKF while also applying the EKF and UKF in both state and joint parameter estimation. Recall from Chapter ~\ref{chap:two} that applying any version of the KF depends on converting systems into their state space format. Chapter ~\ref{chap:three} explores how the KF can be applied to linear dynamical systems while Chapter ~\ref{Extended Kalman Filters} and Chapter ~\ref{Unscented Kalman Filters} explores the EKF and UKF, respectively, as nonlinear forms of the KF. The EKF and UKF each use different ways to linearize the system. Since the EKF linearizes around a single point, utilizing the EKF in complex and high dimensional systems can be ineffective. However, the metabolites example given in Chapter ~\ref{Extended Kalman Filters} has only 4 states and is not highly complex, yielding strong performance in state estimation. The UKF addresses the shortcomings of the EKF by linearizing around multiple points instead of just one. When applying the UKF to the metabolites example, the UKF state estimation had a strange anomaly in State 3, while the other states converged quickly. \\

\noindent The applications of the EKF and UKF in parameter estimation was an important aspect of this research. This thesis utilized joint parameter estimation, which declares an unknown parameter as a separate state to be estimated. It was expected that the UKF would perform better than the EKF because the UKF has a more accurate linearization technique. However, when applied to the metabolites example, performance of the EKF is roughly the same as the performance of the UKF. Tuning the $\alpha, \beta$ and $\kappa$ hyper-parameters of the UKF is likely to yield better results. Future directions for parameter estimation includes looking into alternative methods of parameter estimation. While this thesis only explored joint state estimation, there is a separate dual approach to parameter estimation. In addition, exploration of the Dual Unscented Kalman Filter (DUKF) may be promising. The DUKF was adapted from the UKF and is used to simultaneously estimate states and parameters. \\

\noindent The results of the examples explored in this thesis are overly optimistic and may not reflect model performance of real world systems. This is mainly due to the way the model is initialized. Recall the Van der Pol oscillator from Chapter ~\ref{Unscented Kalman Filters}, where it was observed how the initial state guess, process noise values, and measurement noise values largely influence model performance and convergence. All of the examples in the thesis are initialized with values close to the system's true values and assume low values of measurement and process noise, which are factors that aid faster convergence. In addition, all of the parameter estimation examples operate with the assumption that all states and parameters are measurable--which is not an assumption that should be made of all real world systems. \\ 

\noindent Future directions of this work includes applying the methods explored in this thesis to real world examples, such as the Type 1 Diabetes model. The Type 1 Diabetes model is highly complex and contains 12 states, making this system more complex than the examples considered in this thesis. Furthermore, this model will not be using simulated data; therefore, system measurements will likely be more variable and less predictable. In all, the KF, EKF, and UKF are useful state and parameter estimation methods that can greatly aid in mathematical modeling.