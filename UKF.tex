\chapter{Unscented Kalman Filters}
\label{Unscented Kalman Filters}

The Unscented Kalman Filter (UKF) is another nonlinear version of the Kalman Filter developed to address the shortcomings of the EKF. For instance, as opposed to using the Jacobean to linearly approximate around a single point, the UKF uses the Unscented Transform (UT) to approximate around multiple points, known as sigma points. The UT is a method of approximating probability distributions that have undergone a non linear transformation using limited statistics. The UT involves using these sigma points, which are represented in a Sigma Point Matrix, to represent the normal distribution of the data. The covariance and weights of these sigma points are calculated. These sigma points undergo a non-linear transformation, resulting in a posterior distribution that is not normal \cite{inbook, Wan01theunscented} . We are able to approximate the normal distribution of the posterior distribution using the weights and covariance that were calculated prior to the transformation. This process enables the Kalman Filter to be applied to more complex non linear problems. \\ \\
Unlike the Kalman Filter and the Extended Kalman Filter, the UKF also has a set of parameters. Explanations of each parameter and their default values can be found in the chart below. For the UKF, parameters are necessary for controlling the spread of sigma points. This was not needed for the EKF, since the EKF was only linearizing around the mean.
\textcolor{red}{expand here on how to tune parameters.} \\ \\
The term 'unscented' was arbitrarily coined by the developer of the UKF, Jeffrey Uhlmann. In an interview, he shares:
\begin{displayquote}
"Initially I only referred to it as the “new filter.” Needing a more specific name, people in my lab began referring to it as the “Uhlmann filter,” which obviously isn’t a name that I could use, so I had to come up with an official term. One evening everyone else in the lab was at the Royal Opera House, and as I was working I noticed someone’s deodorant on a desk. The word “unscented” caught my eye as the perfect technical term. At first people in the lab thought it was absurd—which is okay because absurdity is my guiding principle—and that it wouldn’t catch on. My claim was that people simply accept technical terms as technical terms: for example, does anyone think about why a tree is called a tree?"
\end{displayquote}

\section{Unscented Kalman Filter Algorithm}

\begin{enumerate}
    \item The first step is to initialize state vector, $x_0$, and covariance, $P_{x_0}$, which is exactly similar to both the KF and the EKF. Since we know that the $x_0$ is normally distributed, the expectation can be calculated by
    \begin{align*}
    	x_0 = \mathbb{E}[x_0]   = \sum^n_{i = a} x_i p_i = [x_a p_a + x_b p_b + \hdots + x_n p_n]^T,
        %\hat{x}_{0} &= \mathbb{E}[x_{0}] 
    \end{align*}
   where $x_0= [x_a, x_b, \hdots, x_n]^T$, $p_a, p_b, \hdots, p_n$  is the respective probability of obtaining each state variable and $T$ is the transpose. Next initialize the state covariance, by
    \begin{align*}
        P_{x_{0}} &= \mathbb{E}[(x_{0}-\hat{x}_{0})(x_{0}-\hat{x}_{0})^{T}].
    \end{align*}
\noindent We need to initialize these values so the model can begin generating and correcting predictions. \\ \\
    It is critical to choose initial values that are close to the actual values of the data. Failure to do causes the model to converge slowly.
    
        \item Unlike the KF and EKF, the UKF requires the calculation of sigma points, which is a way characterizing the distribution of the data. The number of sigma points is deterministic and depends on the dimensions of the system. In general, a UFK will have  $2 \cdot d_x$ + 1 sigma points, where $d_x$ represents the dimension of the state vector \cite{inbook, inproceedings, Wan01theunscented}.  All sigma points are stored in a sigma point matrix, called $\chi$. To get a general idea about the distribution of the data, we use $\lambda$, which is a scalar value that determines how spread out the sigma points are from the mean. $\lambda$ can be calculated by
         \begin{align*}
        \lambda = \alpha^{2}(d_{x}+\kappa)-d_{x},
         \end{align*}
         where  $\alpha$ and $\kappa$ are both parameters that control for the spread of sigma points around the mean value of the state. The spread of the sigma points is proportional to $\alpha$ . For both $\alpha$ and $\kappa $,  the smaller the values are, the closer the sigma points are to the mean.\\ \\
       Another parameter of the UKF is $\beta$, which uses information regarding state distribution to adjust sigma points. $\beta$ has a default value of 2 if the data is Gaussian (which is an assumption we will make throughout this paper).  \\ \\
       Since the goal of this step is to characterize the distribution, set one of these sigma points to the mean, $\chi_{\ 0,k-1}$ which can be expressed as 
    \begin{align*}
        \chi_{\ 0,k-1} &= x_{k-1} ,
     \end{align*}
     where 0 is a row on $\chi$ and $k-1$ is the time step. Then allocate half of the remaining points will be smaller than the mean and the other half will be larger than the mean by
         \begin{align*}
        \chi_{\ i,k-1} &= \hat{x}_{k-1} +  \bigg(\sqrt{(d_{x}+\lambda )P_{x_{k-1}}}\bigg)_{i} \quad \quad \quad i=1,\dots, d_x, 
        \end{align*}
         \begin{align*}
        \chi_{\ i,k-1} &= \hat{x}_{k-1} - \bigg(\sqrt{(d_{x}+\lambda )P_{x_{k-1}}}\bigg)_{i} \quad \quad \quad i=d_x + 1,\dots,2 d_x + 1,
        \end{align*}
        where $(\sqrt{(d_{x}+\lambda)P_{x_{k-1}}})$ is a matrix, and the $i$ subscript is the $i^{th}$ column of the matrix. 
        Recall that the square root of a matrix, satisfies the following condition: $A = B^2$, where $A$ and $B$ are both matrices. 
        
        
        \item Next, we must calculate a weight for each sigma point. Weights are scalars used to calculate posterior sigma points after they have undergone a nonlinear transformation. One set of weights,  $W^{(m)}$, will be used to calculate the posterior mean while another set of weights $W^{(c)}$ will be used to calculate the posterior covariance. Weights can have positive or negative values, but will ultimately sum to 1 \cite{article6}.
        
        Calculations of the weights at the initial time step follow a formula distinct from the other time steps
         \textcolor{red}{because........}. $W^{(m)}_{0} $ can be found by
            \begin{align*}
        W^{(m)}_{0} = \frac{\lambda}{d_{x}+ \lambda} ,
         \end{align*}
         where the 0 is the time step. Similarly, the weight for the covariance in the initial time step, $W^{(c)}_{0}$, is given by
        \begin{align*}
        W^{(c)}_{0} = \frac{\lambda}{d_{x}+ \lambda} + (1 - \alpha^{2} + \beta) ,
         \end{align*}
          \textcolor{red}{expand why parameters come in here........}. In later time steps, $W^{(m)}_{i} $ and$ W^{(c)}_{i}$ follow the same equation, given by
               \begin{align*}
        W^{(m)}_{i} = W^{(c)}_{i} = \frac{\lambda}{2(d_{x}+ \lambda) } \quad \quad \quad i=1,\dots,2d_{x}.
            \end{align*}
           
          
           
        \item This step performs a nonlinear transformation on the sigma point matrix, $\chi$, in order to generate a prediction and provide an update on the covariance matrix. After $\chi$ undergoes a nonlinear transformation, $f$, the result is a transformed sigma point matrix at time step $k$ given the last time step, $k-1$, which will is called $\chi_{k | k - 1}$ and is given by
        \begin{align*}
        \chi_{k | k - 1} = f(\chi).
        \end{align*}
        
        
        Though $\chi$ has a Gaussian distribution,  $ \chi_{k | k - 1} $ does not because it has been transformed by the nonlinear state function $f$. The sum of the columns in $ \chi_{k | k - 1} $ and the weights calculated in step 3 will then be used to generate a prediction of the state variables, $x_k$, by 
        \textcolor{red}{expand on why this works....}
        \begin{align*}
         x_{k} = \sum^{2d_x}_{i = 0} W_i^{(m)} \chi_{i, k | k - 1}.
        \end{align*}
       Adding the weights help approximate the Gaussian distribution of the state variables after they undergo a transformation. Next, calculate the posterior covariance matrix for the state variable, which is necessary for updating the state covariance later on by

        \begin{align*}
        P_{x, k | k-1} = \sum^{2d_x}_{i = 0} W_i^{(c)} (\chi_{i, k | k - 1} -   x_{k} )(\chi_{i, k | k - 1} - x_{k} )^T + Q,
        \end{align*} 
        where $Q$ is process noise that provides the error in our model $f$, $T$ is the transpose, and $W_i^{(c)}$ are the weights calculated earlier. \\ \\
        Steps 2 through 4 are part of a function known as the Unscented Transform.
        
        
                \item Now that the prediction component of the filter is completed, we move on to the correction step, which begins by transforming our predictions into a format that can be compared with system measurements. Calculation of the posterior (also called augmented) sigma points is necessary for converting system measurements into a format that can be compared with the state variables. System measurements must undergo a non linear transformation, $h$, resulting in a non Gaussian distribution. Therefore, the Unscented Transform is used again. Begin by calculating sigma points, $\chi^{(aug)}$, by
      \begin{align*}
        \chi^{(aug)}_{0, k|k-1} =  x_{k}
        \end{align*}
         \begin{align*}
        \chi^{(aug)}_{ i,k |k-1} &= x_k  \pm \bigg(\sqrt{(d_{x}+\lambda)P_{x_k}} \bigg)_{i} \quad \quad \quad  i=1,\dots,2d_{x}
        \end{align*}
        Recall that $\lambda$ was calculated in step 2. Since $\lambda$ is not time dependent, we can use the same value used earlier. \\ \\
        Now we calculate a sigma point matrix that represents the transformation of the prediction so that it can be compared with the states. This is necessary, especially in cases where measurements are not being obtained for all state variables. This sigma point matrix, $\mathcal{Y}_{k|k-1}$, can be obtained by having the $\chi_{k|k-1}$ undergo nonlinear transformation $h$, by
         \begin{align*}
       \mathcal{Y}_{k|k-1} = h(\chi_{k|k-1}).
       \end{align*}
       $\mathcal{Y}_{k|k-1} $ is a transformed sigma point matrix. While in this format, it cannot be compared with the state variables. However, $\mathcal{Y}_{k|k-1}$ can be used to convert the system measurements into a format, $y_{k} $, which can be found by 
       \begin{align*}
       y_{k} = \sum^{2d_x}_{i = 0} W_i^{(m)}  \mathcal{Y}_{i, k | k - 1}.
       \end{align*}
       Now, the prediction can be compared with actual system measurements.
       
              
       \item The next step is to calculate Kalman Gain in order to determine how much to fix the model. \\ \\
       Unlike previous versions of the KF, in addition to calculating the covariance of the state variables, calculations is also done for the covariance of observations  $P_{y}$ and for state variables with observations $P_{xy}$. When generating the covariance matrix for y the covariance of measurement noise is added. 
        \begin{align*}
       P_{y, k | k-1} = \sum^{2d_x}_{i = 0} W_i^{(c)} (\mathcal{Y}_{i, k | k - 1} -   y_{i, k | k - 1} )(\mathcal{Y}_{i, k | k - 1} -  y_{i, k | k - 1} )^T + R
       \end{align*}
        \begin{align*}
       P_{xy, k | k-1} = \sum^{2d_x}_{i = 0} W_i^{(c)} (\chi^{(aug)}_{i, k | k - 1} -  \hat x_{i, k | k - 1} )(\mathcal{Y}_{i, k | k - 1} -  y_{i, k | k - 1} )^T 
       \end{align*}
       \begin{align*}
       K_k = P_{xy, k | k-1} (P_{y, k | k-1}) ^{-1}
       \end{align*}
        
        
        
      \item The final step includes correcting the prediction and updating the covariance matrix, which are both used in the next iteration.
      \begin{align*}
        x_{k+1} = x_{k} + K_k(\hat y_k - y_{k}),
        \end{align*}
        where $\hat{y}_k$ is system measurements.
        Similar to the KF and EKF, the correction step follows the same equation. 
      
       \begin{align*}
       P_{x, k} = P_{x, k|k-1} -K_k (P_{y, k | k-1} ) {K_k}^T     
       \end{align*}     
  
            
            
\newpage            
\centering
\begin{center}
\begin{tabular}{ |p{2cm}||p{5cm}|p{2cm}| }
    \hline
    \multicolumn{3}{|c|}{Variables in the Unscented Kalman Filter } \\ 
    \hline
    Variable & Description & Dimensions \\
    \hline
    x & Vector containing state variables & $d_x \times 1 $\\ 
    $\chi $& Sigma Point Matrix &$ d_x \times (2 d_x + 1) $\\
    f & Nonlinear state function & $d_x \times d_x $  \\ 
    h & Nonlinear observation function & $d_u \times d_x$\\
    u & Input vector  & $d_u \times 1$\\
    P & Covariance matrix & $d_x \times d_x $  \\
    Q & Process noise covariance & $d_x \times d_x $  \\
    R & Measurement noise covariance & $d_y \times d_y $  \\
     $W^{(m)}$ & Weight for state variable (mean) & scalar \\
    $W^{(c)}$ & Weight for covariance & scalar \\
    \hline
\end{tabular} 
\end{center}

\begin{center}
\begin{tabular}{ |p{1cm}||p{5cm}|p{2cm}| p{1cm}| }
    \hline
    \multicolumn{4}{|c|}{Parameters in the Unscented Kalman Filter } \\ 
    \hline
     & Description & Bounds & Default \\
    \hline
    $\alpha$ & Controls spread of sigma points & $0 < \alpha \leq 1$ & $.001$\\
    $\beta$ & Adjust sigma point weight & $\beta \geq 0$ & 2\\
    $\kappa $ & Sigma point weighting constant & $0 \leq \kappa \leq 3^{*}$  & 0 \\
    \hline
\end{tabular}
\end{center}
* Some use $\kappa  = 3 - d_x$             
            
            
            
            
            
            
            
    
    
        
\end{enumerate}