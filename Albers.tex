\chapter{A Diabetes Example}
\label{A Diabetes Example}

Data assimilation (DA) uses mechanistic and empirical data to train machine learning models to make predictions. In the context of type 2 diabetes, DA was effective in providing recommendations on maintaining certain glucose thresholds, smoothing gaps in sparse finger-prick data, and estimating the state of patient phenotypes through constant estimation of physiological parameters. In this thesis, we explore how DA utilizes Bayesian Inference, a process of gauging the probability of an event, in this case glucose levels, given the known probabilities of events from the empirical data. Accuracy also depends on the estimation of relevant physiological processes; the mechanistic model uses differential equations to represent these processes by monitoring 3 variables (remote insulin, plasma insulin, and glucose) in estimating exchange rates. DA allows empirical and physiological data to work in tandem, resulting prognoses that are not only more accurate than other models, but also personalized for the patient and requiring less data. Further research in implementing DA in biomedicine has the potential to provide better patient care.